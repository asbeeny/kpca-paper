\documentclass{article}

%%% Begin imports
% 
% math
\usepackage{amsmath}
\usepackage{amssymb}
\usepackage{amsfonts}
% 
% theorems
\usepackage{amsthm}
% 
% algorithm / code listing
\usepackage[boxruled]{algorithm2e}
\usepackage{listings}
% 
% graphics
\usepackage{float}
\usepackage{graphicx}
\usepackage{pgfplots}
\pgfplotsset{compat=1.18}
% 
% links and cross-referencing
% these have to be loaded as late as possible, cleveref last
\usepackage{hyperref}
\usepackage[noabbrev]{cleveref}
\crefname{enumi}{part}{parts}
\Crefname{enumi}{Part}{Parts}
%
%%% End imports

%%% Begin environments
% these have to be loaded after cleveref
\newtheorem{theorem}{Theorem}[section]
\newtheorem{lemma}[theorem]{Lemma}
\theoremstyle{definition}
\newtheorem{definition}[theorem]{Definition}
\newtheorem{example}[theorem]{Example}
\newtheorem{remark}[theorem]{Remark}
%%% End environments

%%% Begin macros
% symbols
\def\FF{\mathbb{F}}
\def\RR{\mathbb{R}}
\def\CC{\mathbb{C}}
\def\NN{\mathbb{N}}
\def\QQ{\mathbb{Q}}
\def\hilbert#1{\mathcal{#1}} % define the look of Hilbert spaces
\def\X{\hilbert{X}}
\def\H{\hilbert{H}}
\def\dotprod{\boldsymbol{\cdot}}
\def\MSE{\mathrm{MSE}}

% listing style
\lstset{
    basicstyle=\ttfamily\small,
    commentstyle=\itshape\color{gray},
    numbers=left,
    numberstyle=\tiny
}

% commands
\def\ipt#1{\left\langle #1 \right\rangle}

% operators
\DeclareMathOperator{\lspan}{span}
\DeclareMathOperator{\mean}{mean}
\DeclareMathOperator{\rank}{rank}
\DeclareMathOperator{\rowmean}{row\,mean}
\DeclareMathOperator{\colmean}{col\,mean}
\DeclareMathOperator{\row}{row}
\DeclareMathOperator{\col}{col}
\DeclareMathOperator{\cov}{cov}
\DeclareMathOperator{\corr}{corr}
\DeclareMathOperator{\Var}{var} 
\DeclareMathOperator{\tr}{tr} 
%%% End macros

%%% Set title
\title{Kernel Principal Component Analysis}
\author{Alex Beeny \url{abeeny@siue.edu}}
\date{Spring 2024}

\setcounter{tocdepth}{1}

\begin{document}
\maketitle
\tableofcontents
\begin{abstract}
    Principal component analysis (PCA) and kernel methods are tools often used in data science.
    The underlying theory of these tools depend on the properties of a special type of Hilbert space called a reproducing kernel Hilbert space (RKHS).
    This paper explores the essence of RKHSs using data science examples, in particular, PCA and kernel PCA.
    When kernel methods are applied to PCA, we can analyze nonlinear data in a high-dimensional feature space with some nice properties.
\end{abstract}
\section{Introduction}
\label{sec:introduction}
% TODO: Add intro paragraph
% 
% Pattern recognition is an applied science that draws from a variety of mathematical fields.
% In particular, unsupervised learning techniques, such as principal component analysis, are used to reveal patterns in data.
% From a theoretical perspective, we are interested in 

% % TODO: Decide on regression intro
In linear regression, the equation of a line \(y = a_0 + a_1 x\) is used to model observations based on training data.
Here, the input variable \(x\) is used to predict the response variable \(y\).
The parameters \(a_0\) and \(a_1\) are chosen such that the residual error\footnote{The residual for a given observation \(\left(x^{(i)},y^{(i)}\right)\) is \(\left|y^{(i)} - \left(a_0 + a_1 x^{(i)}\right)\right|\).} is minimized.
It seems natural to model data using polyomial equations in a similar way, that is, determine \(a_0, a_1, \dots, a_n\) such that
\begin{equation}
    \label{eqn:polynomial-regression}
    y = a_0 + a_1 x + a_2 x^2 + \dots + a_n x^n
\end{equation}
minimizes the residual error.

In multiple linear regression, the equation of a hyperplane
\begin{equation}
    \label{eqn:multiple-linear-regression}
    y = a_0 + a_1 x_1 + a_2 x_2 + \dots + a_n x_n
\end{equation}
is used to predict \(y\) using inputs \(x_1, x_2, \dots x_n\).
It follows that these observations are points in \(n + 1\) dimensions.

% \begin{example}
%     \label{eg:regression}
%     % TODO: Intro example
Suppose we have a set of points \(\left\{\left(x^{(i)}, y^{(i)}\right)\right\}_{i=1}^{k}\) in \(\RR^2\).
\begin{figure}
    \centering
    % This file was created by matlab2tikz.
%
%The latest updates can be retrieved from
%  http://www.mathworks.com/matlabcentral/fileexchange/22022-matlab2tikz-matlab2tikz
%where you can also make suggestions and rate matlab2tikz.
%
\definecolor{mycolor1}{rgb}{0.00000,0.44700,0.74100}%
\definecolor{mycolor2}{rgb}{0.85000,0.32500,0.09800}%
%
\begin{tikzpicture}[scale=.5]

\begin{axis}[%
width=2.351in,
height=2.351in,
at={(1.432in,0.692in)},
scale only axis,
xmin=0.8,
xmax=4.2,
ymin=0.899321373319454,
ymax=5.81637386238712,
axis background/.style={fill=white}
]
\addplot [color=mycolor1, only marks, mark=o, mark options={solid, mycolor1}, forget plot]
  table[row sep=crcr]{%
1.03102364618709	1.89635200608252\\
1.05874710866312	1.99081508150367\\
1.07666904385065	1.91861076322039\\
1.16700088292618	1.94476477636428\\
1.15852136521559	1.81443225670274\\
1.2573342402677	1.6672445960115\\
1.32425316409503	1.46193166896814\\
1.35639125095251	1.28539362155645\\
1.426390314927	1.29190594605978\\
1.51116628754242	1.16317920678518\\
1.52771797346067	1.16665342088936\\
1.55800752339787	1.25087120537306\\
1.61924984519265	1.08929287399078\\
1.66276456066876	1.02421642799269\\
1.66055829489539	1.07428789900092\\
1.73750766340902	1.05281355009515\\
1.83674063474356	1.06758706338948\\
1.84869247383361	0.927630367762595\\
1.91747269832344	1.03854250275448\\
1.91980499776142	1.01423324648185\\
2.00193550226933	1.13244229108506\\
2.06800875847557	0.98330814258723\\
2.05915455135406	0.990051396490358\\
2.1604277789588	0.908601490469625\\
2.19454470344622	0.899321373319454\\
2.22181395702176	1.0802522633897\\
2.29887400433542	1.0643767640201\\
2.29311086519133	1.13628838486694\\
2.33616069695452	1.02373787382925\\
2.41469230007855	1.36282093678026\\
2.47028403338547	1.2333901421078\\
2.51480903018198	1.3697316642467\\
2.54823716631414	1.27787197059576\\
2.65864684268995	1.41756546923059\\
2.652174488392	1.49433675309514\\
2.7533065654767	1.62304645873151\\
2.82361200029074	1.56631122687407\\
2.84993319641058	1.56911713696776\\
2.90279326280176	1.70524889840619\\
2.93865528830231	1.73949641810109\\
2.95551971666823	1.91897498779834\\
3.04868544243925	2.05861606463079\\
3.12882475635046	2.23744425724559\\
3.2021473479784	2.30906193331213\\
3.18709381272722	2.48714846308874\\
3.20118031790591	2.48677526727991\\
3.3049904247738	2.69403776037719\\
3.36128797731352	2.95522296897976\\
3.39319148605881	2.85358457032387\\
3.41553263131144	3.04520265961224\\
3.56072997798632	3.4707221841804\\
3.47921429467124	3.22300037122612\\
3.584700838626	3.438352021179\\
3.6103512322757	3.62591511616815\\
3.6809161525102	3.77399094847704\\
3.75953554257179	4.00632071052316\\
3.80414143923349	4.13459951411797\\
3.82867794775566	4.44784460055639\\
3.92331010580159	4.61452734184493\\
3.96867181772516	4.85837734147937\\
4.01942142653548	4.95719769247658\\
};
\addplot [color=mycolor2, forget plot]
  table[row sep=crcr]{%
0.8	2.49166207857362\\
0.85	2.37121960692234\\
0.9	2.25583192010692\\
0.95	2.14549901812737\\
1	2.04022090098368\\
1.05	1.93999756867586\\
1.1	1.84482902120391\\
1.15	1.75471525856781\\
1.2	1.66965628076759\\
1.25	1.58965208780323\\
1.3	1.51470267967473\\
1.35	1.4448080563821\\
1.4	1.37996821792533\\
1.45	1.32018316430443\\
1.5	1.26545289551939\\
1.55	1.21577741157022\\
1.6	1.17115671245692\\
1.65	1.13159079817948\\
1.7	1.0970796687379\\
1.75	1.06762332413219\\
1.8	1.04322176436235\\
1.85	1.02387498942837\\
1.9	1.00958299933025\\
1.95	1.000345794068\\
2	0.996163373641616\\
2.05	0.997035738051095\\
2.1	1.00296288729644\\
2.15	1.01394482137765\\
2.2	1.02998154029473\\
2.25	1.05107304404767\\
2.3	1.07721933263647\\
2.35	1.10842040606114\\
2.4	1.14467626432168\\
2.45	1.18598690741808\\
2.5	1.23235233535035\\
2.55	1.28377254811848\\
2.6	1.34024754572247\\
2.65	1.40177732816233\\
2.7	1.46836189543806\\
2.75	1.54000124754965\\
2.8	1.61669538449711\\
2.85	1.69844430628043\\
2.9	1.78524801289962\\
2.95	1.87710650435467\\
3	1.97401978064558\\
3.05	2.07598784177237\\
3.1	2.18301068773501\\
3.15	2.29508831853353\\
3.2	2.4122207341679\\
3.25	2.53440793463815\\
3.3	2.66164991994425\\
3.35	2.79394669008623\\
3.4	2.93129824506406\\
3.45	3.07370458487777\\
3.5	3.22116570952733\\
3.55	3.37368161901277\\
3.6	3.53125231333406\\
3.65	3.69387779249123\\
3.7	3.86155805648426\\
3.75	4.03429310531315\\
3.8	4.21208293897791\\
3.85	4.39492755747853\\
3.9	4.58282696081502\\
3.95	4.77578114898737\\
4	4.97379012199559\\
4.05	5.17685387983967\\
4.1	5.38497242251963\\
4.15	5.59814575003544\\
4.2	5.81637386238712\\
};
\end{axis}

\begin{axis}[%
width=2.351in,
height=2.351in,
at={(4.526in,0.692in)},
scale only axis,
xmin=0,
xmax=6,
ymin=0,
ymax=6,
axis background/.style={fill=white}
]
\addplot [color=mycolor1, only marks, mark=o, mark options={solid, mycolor1}, forget plot]
  table[row sep=crcr]{%
1.93891517424856	1.89635200608252\\
1.88595700545003	1.99081508150367\\
1.85254005458367	1.91861076322039\\
1.69388752904577	1.94476477636428\\
1.70808629279864	1.81443225670274\\
1.55155243067876	1.6672445960115\\
1.45663378623558	1.46193166896814\\
1.41423222185048	1.28539362155645\\
1.32902807080954	1.29190594605978\\
1.23895839843506	1.16317920678518\\
1.2230503125921	1.16665342088936\\
1.19535734937288	1.25087120537306\\
1.14497068038582	1.08929287399078\\
1.11372774154094	1.02421642799269\\
1.11522067116432	1.07428789900092\\
1.06890222676899	1.05281355009515\\
1.02665362034394	1.06758706338948\\
1.02289396747459	0.927630367762595\\
1.00681075552201	1.03854250275448\\
1.00643123838405	1.01423324648185\\
1.00000374616903	1.13244229108506\\
1.00462519122939	0.98330814258723\\
1.0034992609459	0.990051396490358\\
1.02573707226165	0.908601490469625\\
1.03784764163898	0.899321373319454\\
1.04920143152965	1.0802522633897\\
1.08932567046749	1.0643767640201\\
1.08591397929321	1.13628838486694\\
1.11300401417695	1.02373787382925\\
1.17196970374444	1.36282093678026\\
1.22116707205731	1.2333901421078\\
1.26502833755691	1.3697316642467\\
1.30056399052816	1.27787197059576\\
1.43381566338544	1.41756546923059\\
1.42533156330936	1.49433675309514\\
1.5674707815903	1.62304645873151\\
1.67833672702291	1.56631122687407\\
1.72238643836071	1.56911713696776\\
1.81503567536025	1.70524889840619\\
1.88107375025789	1.73949641810109\\
1.91301792894173	1.91897498779834\\
2.09974115718401	2.05861606463079\\
2.27424533054968	2.23744425724559\\
2.4451582462515	2.30906193331213\\
2.40919172021524	2.48714846308874\\
2.44283415612453	2.48677526727991\\
2.70300000875131	2.69403776037719\\
2.85310495717834	2.95522296897976\\
2.94098251682676	2.85358457032387\\
3.00373263030748	3.04520265961224\\
3.43587806418517	3.4707221841804\\
3.18807492955975	3.22300037122612\\
3.51127674794195	3.438352021179\\
3.59323109129186	3.62591511616815\\
3.82547911176969	3.77399094847704\\
4.09596532557341	4.00632071052316\\
4.25492633275947	4.13459951411797\\
4.34406303660786	4.44784460055639\\
4.69912176307854	4.61452734184493\\
4.87566872590529	4.85837734147937\\
5.0780628979506	4.95719769247658\\
};
\addplot [color=mycolor2, forget plot]
  table[row sep=crcr]{%
0	0.00977935819441985\\
0.05	0.0593099862633467\\
0.1	0.108840614332273\\
0.15	0.1583712424012\\
0.2	0.207901870470127\\
0.25	0.257432498539054\\
0.3	0.306963126607981\\
0.35	0.356493754676908\\
0.4	0.406024382745834\\
0.45	0.455555010814761\\
0.5	0.505085638883688\\
0.55	0.554616266952615\\
0.6	0.604146895021542\\
0.65	0.653677523090468\\
0.7	0.703208151159395\\
0.75	0.752738779228322\\
0.8	0.802269407297249\\
0.85	0.851800035366176\\
0.9	0.901330663435102\\
0.95	0.950861291504029\\
1	1.00039191957296\\
1.05	1.04992254764188\\
1.1	1.09945317571081\\
1.15	1.14898380377974\\
1.2	1.19851443184866\\
1.25	1.24804505991759\\
1.3	1.29757568798652\\
1.35	1.34710631605544\\
1.4	1.39663694412437\\
1.45	1.4461675721933\\
1.5	1.49569820026222\\
1.55	1.54522882833115\\
1.6	1.59475945640008\\
1.65	1.644290084469\\
1.7	1.69382071253793\\
1.75	1.74335134060686\\
1.8	1.79288196867579\\
1.85	1.84241259674471\\
1.9	1.89194322481364\\
1.95	1.94147385288257\\
2	1.99100448095149\\
2.05	2.04053510902042\\
2.1	2.09006573708935\\
2.15	2.13959636515827\\
2.2	2.1891269932272\\
2.25	2.23865762129613\\
2.3	2.28818824936505\\
2.35	2.33771887743398\\
2.4	2.38724950550291\\
2.45	2.43678013357183\\
2.5	2.48631076164076\\
2.55	2.53584138970969\\
2.6	2.58537201777861\\
2.65	2.63490264584754\\
2.7	2.68443327391647\\
2.75	2.73396390198539\\
2.8	2.78349453005432\\
2.85	2.83302515812325\\
2.9	2.88255578619217\\
2.95	2.9320864142611\\
3	2.98161704233003\\
3.05	3.03114767039895\\
3.1	3.08067829846788\\
3.15	3.13020892653681\\
3.2	3.17973955460574\\
3.25	3.22927018267466\\
3.3	3.27880081074359\\
3.35	3.32833143881252\\
3.4	3.37786206688144\\
3.45	3.42739269495037\\
3.5	3.4769233230193\\
3.55	3.52645395108822\\
3.6	3.57598457915715\\
3.65	3.62551520722608\\
3.7	3.675045835295\\
3.75	3.72457646336393\\
3.8	3.77410709143286\\
3.85	3.82363771950178\\
3.9	3.87316834757071\\
3.95	3.92269897563964\\
4	3.97222960370856\\
4.05	4.02176023177749\\
4.1	4.07129085984642\\
4.15	4.12082148791534\\
4.2	4.17035211598427\\
4.25	4.2198827440532\\
4.3	4.26941337212213\\
4.35	4.31894400019105\\
4.4	4.36847462825998\\
4.45	4.41800525632891\\
4.5	4.46753588439783\\
4.55	4.51706651246676\\
4.6	4.56659714053569\\
4.65	4.61612776860461\\
4.7	4.66565839667354\\
4.75	4.71518902474247\\
4.8	4.76471965281139\\
4.85	4.81425028088032\\
4.9	4.86378090894925\\
4.95	4.91331153701817\\
5	4.9628421650871\\
5.05	5.01237279315603\\
5.1	5.06190342122495\\
5.15	5.11143404929388\\
5.2	5.16096467736281\\
5.25	5.21049530543173\\
5.3	5.26002593350066\\
5.35	5.30955656156959\\
5.4	5.35908718963852\\
5.45	5.40861781770744\\
5.5	5.45814844577637\\
5.55	5.5076790738453\\
5.6	5.55720970191422\\
5.65	5.60674032998315\\
5.7	5.65627095805208\\
5.75	5.705801586121\\
5.8	5.75533221418993\\
5.85	5.80486284225886\\
5.9	5.85439347032778\\
5.95	5.90392409839671\\
6	5.95345472646564\\
};
\end{axis}

\begin{axis}[%
width=2.351in,
height=2.711in,
at={(7.619in,0.512in)},
scale only axis,
plot box ratio=1 1 1,
xmin=1,
xmax=4.01942142653548,
tick align=outside,
ymin=0,
ymax=16.1557486040925,
zmin=-10.1905643855997,
zmax=20,
view={-16.2617026369184}{24.9729119273008},
axis background/.style={fill=white},
axis x line*=bottom,
axis y line*=left,
axis z line*=left
]
\addplot3 [color=mycolor1, only marks, mark=o, mark options={solid, mycolor1}]
 table[row sep=crcr] {%
1.03102364618709	1.06300975899692	1.89635200608252\\
1.05874710866312	1.12094544010253	1.99081508150367\\
1.07666904385065	1.15921622998628	1.91861076322039\\
1.16700088292618	1.36189106075048	1.94476477636428\\
1.15852136521559	1.34217175366099	1.81443225670274\\
1.2573342402677	1.58088939174955	1.6672445960115\\
1.32425316409503	1.7536464426157	1.46193166896814\\
1.35639125095251	1.83979722566052	1.28539362155645\\
1.426390314927	2.03458933051756	1.29190594605978\\
1.51116628754242	2.28362354860474	1.16317920678518\\
1.52771797346067	2.33392220643477	1.16665342088936\\
1.55800752339787	2.42738744296438	1.25087120537306\\
1.61924984519265	2.62197006115643	1.08929287399078\\
1.66276456066876	2.76478598421597	1.02421642799269\\
1.66055829489539	2.7574538507459	1.07428789900092\\
1.73750766340902	3.01893288040506	1.05281355009515\\
1.83674063474356	3.37361615931818	1.06758706338948\\
1.84869247383361	3.41766386280903	0.927630367762595\\
1.91747269832344	3.67670154881577	1.03854250275448\\
1.91980499776142	3.68565122942973	1.01423324648185\\
2.00193550226933	4.00774575524637	1.13244229108506\\
2.06800875847557	4.27666022513168	0.98330814258723\\
2.05915455135406	4.24011746636215	0.990051396490358\\
2.1604277789588	4.66744818809686	0.908601490469625\\
2.19454470344622	4.81602645542386	0.899321373319454\\
2.22181395702176	4.93645725961667	1.0802522633897\\
2.29887400433542	5.28482168780919	1.0643767640201\\
2.29311086519133	5.25835744005851	1.13628838486694\\
2.33616069695452	5.45764680199503	1.02373787382925\\
2.41469230007855	5.83073890405865	1.36282093678026\\
2.47028403338547	6.10230320559921	1.2333901421078\\
2.51480903018198	6.32426445828482	1.3697316642467\\
2.54823716631414	6.49351265578471	1.27787197059576\\
2.65864684268995	7.06840303414524	1.41756546923059\\
2.652174488392	7.03402951687734	1.49433675309514\\
2.7533065654767	7.58069704349711	1.62304645873151\\
2.82361200029074	7.97278472818587	1.56631122687407\\
2.84993319641058	8.12211922400305	1.56911713696776\\
2.90279326280176	8.42620872656731	1.70524889840619\\
2.93865528830231	8.63569490346712	1.73949641810109\\
2.95551971666823	8.73509679561465	1.91897498779834\\
3.04868544243925	9.29448292694101	2.05861606463079\\
3.12882475635046	9.78954435595154	2.23744425724559\\
3.2021473479784	10.2537476381651	2.30906193331213\\
3.18709381272722	10.1575669711241	2.48714846308874\\
3.20118031790591	10.2475554277482	2.48677526727991\\
3.3049904247738	10.9229617078465	2.69403776037719\\
3.36128797731352	11.2982568664324	2.95522296897976\\
3.39319148605881	11.513748461062	2.85358457032387\\
3.41553263131144	11.6658631555532	3.04520265961224\\
3.56072997798632	12.6787979761304	3.4707221841804\\
3.47921429467124	12.1049321082447	3.22300037122612\\
3.584700838626	12.850080102446	3.438352021179\\
3.6103512322757	13.0346360203946	3.62591511616815\\
3.6809161525102	13.5491437218105	3.77399094847704\\
3.75953554257179	14.1341074958606	4.00632071052316\\
3.80414143923349	14.4714920896934	4.13459951411797\\
3.82867794775566	14.6587748276305	4.44784460055639\\
3.92331010580159	15.3923621862849	4.61452734184493\\
3.96867181772516	15.7503559968059	4.85837734147937\\
4.01942142653548	16.1557486040925	4.95719769247658\\
};
 
\addplot3[%
surf,
fill opacity=0.2, draw opacity=0.2, fill=mycolor2, faceted color=black, z buffer=sort, colormap={mymap}{[1pt] rgb(0pt)=(0.2422,0.1504,0.6603); rgb(1pt)=(0.2444,0.1534,0.6728); rgb(2pt)=(0.2464,0.1569,0.6847); rgb(3pt)=(0.2484,0.1607,0.6961); rgb(4pt)=(0.2503,0.1648,0.7071); rgb(5pt)=(0.2522,0.1689,0.7179); rgb(6pt)=(0.254,0.1732,0.7286); rgb(7pt)=(0.2558,0.1773,0.7393); rgb(8pt)=(0.2576,0.1814,0.7501); rgb(9pt)=(0.2594,0.1854,0.761); rgb(11pt)=(0.2628,0.1932,0.7828); rgb(12pt)=(0.2645,0.1972,0.7937); rgb(13pt)=(0.2661,0.2011,0.8043); rgb(14pt)=(0.2676,0.2052,0.8148); rgb(15pt)=(0.2691,0.2094,0.8249); rgb(16pt)=(0.2704,0.2138,0.8346); rgb(17pt)=(0.2717,0.2184,0.8439); rgb(18pt)=(0.2729,0.2231,0.8528); rgb(19pt)=(0.274,0.228,0.8612); rgb(20pt)=(0.2749,0.233,0.8692); rgb(21pt)=(0.2758,0.2382,0.8767); rgb(22pt)=(0.2766,0.2435,0.884); rgb(23pt)=(0.2774,0.2489,0.8908); rgb(24pt)=(0.2781,0.2543,0.8973); rgb(25pt)=(0.2788,0.2598,0.9035); rgb(26pt)=(0.2794,0.2653,0.9094); rgb(27pt)=(0.2798,0.2708,0.915); rgb(28pt)=(0.2802,0.2764,0.9204); rgb(29pt)=(0.2806,0.2819,0.9255); rgb(30pt)=(0.2809,0.2875,0.9305); rgb(31pt)=(0.2811,0.293,0.9352); rgb(32pt)=(0.2813,0.2985,0.9397); rgb(33pt)=(0.2814,0.304,0.9441); rgb(34pt)=(0.2814,0.3095,0.9483); rgb(35pt)=(0.2813,0.315,0.9524); rgb(36pt)=(0.2811,0.3204,0.9563); rgb(37pt)=(0.2809,0.3259,0.96); rgb(38pt)=(0.2807,0.3313,0.9636); rgb(39pt)=(0.2803,0.3367,0.967); rgb(40pt)=(0.2798,0.3421,0.9702); rgb(41pt)=(0.2791,0.3475,0.9733); rgb(42pt)=(0.2784,0.3529,0.9763); rgb(43pt)=(0.2776,0.3583,0.9791); rgb(44pt)=(0.2766,0.3638,0.9817); rgb(45pt)=(0.2754,0.3693,0.984); rgb(46pt)=(0.2741,0.3748,0.9862); rgb(47pt)=(0.2726,0.3804,0.9881); rgb(48pt)=(0.271,0.386,0.9898); rgb(49pt)=(0.2691,0.3916,0.9912); rgb(50pt)=(0.267,0.3973,0.9924); rgb(51pt)=(0.2647,0.403,0.9935); rgb(52pt)=(0.2621,0.4088,0.9946); rgb(53pt)=(0.2591,0.4145,0.9955); rgb(54pt)=(0.2556,0.4203,0.9965); rgb(55pt)=(0.2517,0.4261,0.9974); rgb(56pt)=(0.2473,0.4319,0.9983); rgb(57pt)=(0.2424,0.4378,0.9991); rgb(58pt)=(0.2369,0.4437,0.9996); rgb(59pt)=(0.2311,0.4497,0.9995); rgb(60pt)=(0.225,0.4559,0.9985); rgb(61pt)=(0.2189,0.462,0.9968); rgb(62pt)=(0.2128,0.4682,0.9948); rgb(63pt)=(0.2066,0.4743,0.9926); rgb(64pt)=(0.2006,0.4803,0.9906); rgb(65pt)=(0.195,0.4861,0.9887); rgb(66pt)=(0.1903,0.4919,0.9867); rgb(67pt)=(0.1869,0.4975,0.9844); rgb(68pt)=(0.1847,0.503,0.9819); rgb(69pt)=(0.1831,0.5084,0.9793); rgb(70pt)=(0.1818,0.5138,0.9766); rgb(71pt)=(0.1806,0.5191,0.9738); rgb(72pt)=(0.1795,0.5244,0.9709); rgb(73pt)=(0.1785,0.5296,0.9677); rgb(74pt)=(0.1778,0.5349,0.9641); rgb(75pt)=(0.1773,0.5401,0.9602); rgb(76pt)=(0.1768,0.5452,0.956); rgb(77pt)=(0.1764,0.5504,0.9516); rgb(78pt)=(0.1755,0.5554,0.9473); rgb(79pt)=(0.174,0.5605,0.9432); rgb(80pt)=(0.1716,0.5655,0.9393); rgb(81pt)=(0.1686,0.5705,0.9357); rgb(82pt)=(0.1649,0.5755,0.9323); rgb(83pt)=(0.161,0.5805,0.9289); rgb(84pt)=(0.1573,0.5854,0.9254); rgb(85pt)=(0.154,0.5902,0.9218); rgb(86pt)=(0.1513,0.595,0.9182); rgb(87pt)=(0.1492,0.5997,0.9147); rgb(88pt)=(0.1475,0.6043,0.9113); rgb(89pt)=(0.1461,0.6089,0.908); rgb(90pt)=(0.1446,0.6135,0.905); rgb(91pt)=(0.1429,0.618,0.9022); rgb(92pt)=(0.1408,0.6226,0.8998); rgb(93pt)=(0.1383,0.6272,0.8975); rgb(94pt)=(0.1354,0.6317,0.8953); rgb(95pt)=(0.1321,0.6363,0.8932); rgb(96pt)=(0.1288,0.6408,0.891); rgb(97pt)=(0.1253,0.6453,0.8887); rgb(98pt)=(0.1219,0.6497,0.8862); rgb(99pt)=(0.1185,0.6541,0.8834); rgb(100pt)=(0.1152,0.6584,0.8804); rgb(101pt)=(0.1119,0.6627,0.877); rgb(102pt)=(0.1085,0.6669,0.8734); rgb(103pt)=(0.1048,0.671,0.8695); rgb(104pt)=(0.1009,0.675,0.8653); rgb(105pt)=(0.0964,0.6789,0.8609); rgb(106pt)=(0.0914,0.6828,0.8562); rgb(107pt)=(0.0855,0.6865,0.8513); rgb(108pt)=(0.0789,0.6902,0.8462); rgb(109pt)=(0.0713,0.6938,0.8409); rgb(110pt)=(0.0628,0.6972,0.8355); rgb(111pt)=(0.0535,0.7006,0.8299); rgb(112pt)=(0.0433,0.7039,0.8242); rgb(113pt)=(0.0328,0.7071,0.8183); rgb(114pt)=(0.0234,0.7103,0.8124); rgb(115pt)=(0.0155,0.7133,0.8064); rgb(116pt)=(0.0091,0.7163,0.8003); rgb(117pt)=(0.0046,0.7192,0.7941); rgb(118pt)=(0.0019,0.722,0.7878); rgb(119pt)=(0.0009,0.7248,0.7815); rgb(120pt)=(0.0018,0.7275,0.7752); rgb(121pt)=(0.0046,0.7301,0.7688); rgb(122pt)=(0.0094,0.7327,0.7623); rgb(123pt)=(0.0162,0.7352,0.7558); rgb(124pt)=(0.0253,0.7376,0.7492); rgb(125pt)=(0.0369,0.74,0.7426); rgb(126pt)=(0.0504,0.7423,0.7359); rgb(127pt)=(0.0638,0.7446,0.7292); rgb(128pt)=(0.077,0.7468,0.7224); rgb(129pt)=(0.0899,0.7489,0.7156); rgb(130pt)=(0.1023,0.751,0.7088); rgb(131pt)=(0.1141,0.7531,0.7019); rgb(132pt)=(0.1252,0.7552,0.695); rgb(133pt)=(0.1354,0.7572,0.6881); rgb(134pt)=(0.1448,0.7593,0.6812); rgb(135pt)=(0.1532,0.7614,0.6741); rgb(136pt)=(0.1609,0.7635,0.6671); rgb(137pt)=(0.1678,0.7656,0.6599); rgb(138pt)=(0.1741,0.7678,0.6527); rgb(139pt)=(0.1799,0.7699,0.6454); rgb(140pt)=(0.1853,0.7721,0.6379); rgb(141pt)=(0.1905,0.7743,0.6303); rgb(142pt)=(0.1954,0.7765,0.6225); rgb(143pt)=(0.2003,0.7787,0.6146); rgb(144pt)=(0.2061,0.7808,0.6065); rgb(145pt)=(0.2118,0.7828,0.5983); rgb(146pt)=(0.2178,0.7849,0.5899); rgb(147pt)=(0.2244,0.7869,0.5813); rgb(148pt)=(0.2318,0.7887,0.5725); rgb(149pt)=(0.2401,0.7905,0.5636); rgb(150pt)=(0.2491,0.7922,0.5546); rgb(151pt)=(0.2589,0.7937,0.5454); rgb(152pt)=(0.2695,0.7951,0.536); rgb(153pt)=(0.2809,0.7964,0.5266); rgb(154pt)=(0.2929,0.7975,0.517); rgb(155pt)=(0.3052,0.7985,0.5074); rgb(156pt)=(0.3176,0.7994,0.4975); rgb(157pt)=(0.3301,0.8002,0.4876); rgb(158pt)=(0.3424,0.8009,0.4774); rgb(159pt)=(0.3548,0.8016,0.4669); rgb(160pt)=(0.3671,0.8021,0.4563); rgb(161pt)=(0.3795,0.8026,0.4454); rgb(162pt)=(0.3921,0.8029,0.4344); rgb(163pt)=(0.405,0.8031,0.4233); rgb(164pt)=(0.4184,0.803,0.4122); rgb(165pt)=(0.4322,0.8028,0.4013); rgb(166pt)=(0.4463,0.8024,0.3904); rgb(167pt)=(0.4608,0.8018,0.3797); rgb(168pt)=(0.4753,0.8011,0.3691); rgb(169pt)=(0.4899,0.8002,0.3586); rgb(170pt)=(0.5044,0.7993,0.348); rgb(171pt)=(0.5187,0.7982,0.3374); rgb(172pt)=(0.5329,0.797,0.3267); rgb(173pt)=(0.547,0.7957,0.3159); rgb(175pt)=(0.5748,0.7929,0.2941); rgb(176pt)=(0.5886,0.7913,0.2833); rgb(177pt)=(0.6024,0.7896,0.2726); rgb(178pt)=(0.6161,0.7878,0.2622); rgb(179pt)=(0.6297,0.7859,0.2521); rgb(180pt)=(0.6433,0.7839,0.2423); rgb(181pt)=(0.6567,0.7818,0.2329); rgb(182pt)=(0.6701,0.7796,0.2239); rgb(183pt)=(0.6833,0.7773,0.2155); rgb(184pt)=(0.6963,0.775,0.2075); rgb(185pt)=(0.7091,0.7727,0.1998); rgb(186pt)=(0.7218,0.7703,0.1924); rgb(187pt)=(0.7344,0.7679,0.1852); rgb(188pt)=(0.7468,0.7654,0.1782); rgb(189pt)=(0.759,0.7629,0.1717); rgb(190pt)=(0.771,0.7604,0.1658); rgb(191pt)=(0.7829,0.7579,0.1608); rgb(192pt)=(0.7945,0.7554,0.157); rgb(193pt)=(0.806,0.7529,0.1546); rgb(194pt)=(0.8172,0.7505,0.1535); rgb(195pt)=(0.8281,0.7481,0.1536); rgb(196pt)=(0.8389,0.7457,0.1546); rgb(197pt)=(0.8495,0.7435,0.1564); rgb(198pt)=(0.86,0.7413,0.1587); rgb(199pt)=(0.8703,0.7392,0.1615); rgb(200pt)=(0.8804,0.7372,0.165); rgb(201pt)=(0.8903,0.7353,0.1695); rgb(202pt)=(0.9,0.7336,0.1749); rgb(203pt)=(0.9093,0.7321,0.1815); rgb(204pt)=(0.9184,0.7308,0.189); rgb(205pt)=(0.9272,0.7298,0.1973); rgb(206pt)=(0.9357,0.729,0.2061); rgb(207pt)=(0.944,0.7285,0.2151); rgb(208pt)=(0.9523,0.7284,0.2237); rgb(209pt)=(0.9606,0.7285,0.2312); rgb(210pt)=(0.9689,0.7292,0.2373); rgb(211pt)=(0.977,0.7304,0.2418); rgb(212pt)=(0.9842,0.733,0.2446); rgb(213pt)=(0.99,0.7365,0.2429); rgb(214pt)=(0.9946,0.7407,0.2394); rgb(215pt)=(0.9966,0.7458,0.2351); rgb(216pt)=(0.9971,0.7513,0.2309); rgb(217pt)=(0.9972,0.7569,0.2267); rgb(218pt)=(0.9971,0.7626,0.2224); rgb(219pt)=(0.9969,0.7683,0.2181); rgb(220pt)=(0.9966,0.774,0.2138); rgb(221pt)=(0.9962,0.7798,0.2095); rgb(222pt)=(0.9957,0.7856,0.2053); rgb(223pt)=(0.9949,0.7915,0.2012); rgb(224pt)=(0.9938,0.7974,0.1974); rgb(225pt)=(0.9923,0.8034,0.1939); rgb(226pt)=(0.9906,0.8095,0.1906); rgb(227pt)=(0.9885,0.8156,0.1875); rgb(228pt)=(0.9861,0.8218,0.1846); rgb(229pt)=(0.9835,0.828,0.1817); rgb(230pt)=(0.9807,0.8342,0.1787); rgb(231pt)=(0.9778,0.8404,0.1757); rgb(232pt)=(0.9748,0.8467,0.1726); rgb(233pt)=(0.972,0.8529,0.1695); rgb(234pt)=(0.9694,0.8591,0.1665); rgb(235pt)=(0.9671,0.8654,0.1636); rgb(236pt)=(0.9651,0.8716,0.1608); rgb(237pt)=(0.9634,0.8778,0.1582); rgb(238pt)=(0.9619,0.884,0.1557); rgb(239pt)=(0.9608,0.8902,0.1532); rgb(240pt)=(0.9601,0.8963,0.1507); rgb(241pt)=(0.9596,0.9023,0.148); rgb(242pt)=(0.9595,0.9084,0.145); rgb(243pt)=(0.9597,0.9143,0.1418); rgb(244pt)=(0.9601,0.9203,0.1382); rgb(245pt)=(0.9608,0.9262,0.1344); rgb(246pt)=(0.9618,0.932,0.1304); rgb(247pt)=(0.9629,0.9379,0.1261); rgb(248pt)=(0.9642,0.9437,0.1216); rgb(249pt)=(0.9657,0.9494,0.1168); rgb(250pt)=(0.9674,0.9552,0.1116); rgb(251pt)=(0.9692,0.9609,0.1061); rgb(252pt)=(0.9711,0.9667,0.1001); rgb(253pt)=(0.973,0.9724,0.0938); rgb(254pt)=(0.9749,0.9782,0.0872); rgb(255pt)=(0.9769,0.9839,0.0805)}, mesh/rows=13]
table[row sep=crcr, point meta=\thisrow{c}] {%
%
x	y	z	c\\
1	1	2.04022090098368	2.04022090098368\\
1	2	3.0511778681567	3.0511778681567\\
1	3	4.06213483532972	4.06213483532972\\
1	4	5.07309180250274	5.07309180250274\\
1	5	6.08404876967576	6.08404876967576\\
1	6	7.09500573684878	7.09500573684878\\
1	7	8.10596270402179	8.10596270402179\\
1	8	9.11691967119481	9.11691967119481\\
1	9	10.1278766383678	10.1278766383678\\
1	10	11.1388336055408	11.1388336055408\\
1	11	12.1497905727139	12.1497905727139\\
1	12	13.1607475398869	13.1607475398869\\
1	13	14.1717045070599	14.1717045070599\\
1	14	15.1826614742329	15.1826614742329\\
1	15	16.1936184414059	16.1936184414059\\
1	16	17.204575408579	17.204575408579\\
1.25	1	1.0209887937684	1.0209887937684\\
1.25	2	2.03194576094142	2.03194576094142\\
1.25	3	3.04290272811444	3.04290272811444\\
1.25	4	4.05385969528746	4.05385969528746\\
1.25	5	5.06481666246048	5.06481666246048\\
1.25	6	6.07577362963349	6.07577362963349\\
1.25	7	7.08673059680651	7.08673059680651\\
1.25	8	8.09768756397953	8.09768756397953\\
1.25	9	9.10864453115255	9.10864453115255\\
1.25	10	10.1196014983256	10.1196014983256\\
1.25	11	11.1305584654986	11.1305584654986\\
1.25	12	12.1415154326716	12.1415154326716\\
1.25	13	13.1524723998446	13.1524723998446\\
1.25	14	14.1634293670176	14.1634293670176\\
1.25	15	15.1743863341907	15.1743863341907\\
1.25	16	16.1853433013637	16.1853433013637\\
1.5	1	0.00175668655312133	0.00175668655312133\\
1.5	2	1.01271365372614	1.01271365372614\\
1.5	3	2.02367062089916	2.02367062089916\\
1.5	4	3.03462758807218	3.03462758807218\\
1.5	5	4.04558455524519	4.04558455524519\\
1.5	6	5.05654152241821	5.05654152241821\\
1.5	7	6.06749848959123	6.06749848959123\\
1.5	8	7.07845545676425	7.07845545676425\\
1.5	9	8.08941242393727	8.08941242393727\\
1.5	10	9.10036939111029	9.10036939111029\\
1.5	11	10.1113263582833	10.1113263582833\\
1.5	12	11.1222833254563	11.1222833254563\\
1.5	13	12.1332402926293	12.1332402926293\\
1.5	14	13.1441972598024	13.1441972598024\\
1.5	15	14.1551542269754	14.1551542269754\\
1.5	16	15.1661111941484	15.1661111941484\\
1.75	1	-1.01747542066216	-1.01747542066216\\
1.75	2	-0.00651845348914115	-0.00651845348914115\\
1.75	3	1.00443851368388	1.00443851368388\\
1.75	4	2.0153954808569	2.0153954808569\\
1.75	5	3.02635244802991	3.02635244802991\\
1.75	6	4.03730941520293	4.03730941520293\\
1.75	7	5.04826638237595	5.04826638237595\\
1.75	8	6.05922334954897	6.05922334954897\\
1.75	9	7.07018031672199	7.07018031672199\\
1.75	10	8.08113728389501	8.08113728389501\\
1.75	11	9.09209425106802	9.09209425106802\\
1.75	12	10.103051218241	10.103051218241\\
1.75	13	11.1140081854141	11.1140081854141\\
1.75	14	12.1249651525871	12.1249651525871\\
1.75	15	13.1359221197601	13.1359221197601\\
1.75	16	14.1468790869331	14.1468790869331\\
2	1	-2.03670752787744	-2.03670752787744\\
2	2	-1.02575056070442	-1.02575056070442\\
2	3	-0.0147935935314036	-0.0147935935314036\\
2	4	0.996163373641615	0.996163373641615\\
2	5	2.00712034081463	2.00712034081463\\
2	6	3.01807730798765	3.01807730798765\\
2	7	4.02903427516067	4.02903427516067\\
2	8	5.03999124233369	5.03999124233369\\
2	9	6.05094820950671	6.05094820950671\\
2	10	7.06190517667972	7.06190517667972\\
2	11	8.07286214385274	8.07286214385274\\
2	12	9.08381911102576	9.08381911102576\\
2	13	10.0947760781988	10.0947760781988\\
2	14	11.1057330453718	11.1057330453718\\
2	15	12.1166900125448	12.1166900125448\\
2	16	13.1276469797178	13.1276469797178\\
2.25	1	-3.05593963509272	-3.05593963509272\\
2.25	2	-2.0449826679197	-2.0449826679197\\
2.25	3	-1.03402570074668	-1.03402570074668\\
2.25	4	-0.0230687335736661	-0.0230687335736661\\
2.25	5	0.987888233599352	0.987888233599352\\
2.25	6	1.99884520077237	1.99884520077237\\
2.25	7	3.00980216794539	3.00980216794539\\
2.25	8	4.02075913511841	4.02075913511841\\
2.25	9	5.03171610229143	5.03171610229143\\
2.25	10	6.04267306946444	6.04267306946444\\
2.25	11	7.05363003663746	7.05363003663746\\
2.25	12	8.06458700381048	8.06458700381048\\
2.25	13	9.0755439709835	9.0755439709835\\
2.25	14	10.0865009381565	10.0865009381565\\
2.25	15	11.0974579053295	11.0974579053295\\
2.25	16	12.1084148725026	12.1084148725026\\
2.5	1	-4.075171742308	-4.075171742308\\
2.5	2	-3.06421477513498	-3.06421477513498\\
2.5	3	-2.05325780796197	-2.05325780796197\\
2.5	4	-1.04230084078895	-1.04230084078895\\
2.5	5	-0.0313438736159286	-0.0313438736159286\\
2.5	6	0.97961309355709	0.97961309355709\\
2.5	7	1.99057006073011	1.99057006073011\\
2.5	8	3.00152702790313	3.00152702790313\\
2.5	9	4.01248399507615	4.01248399507615\\
2.5	10	5.02344096224916	5.02344096224916\\
2.5	11	6.03439792942218	6.03439792942218\\
2.5	12	7.0453548965952	7.0453548965952\\
2.5	13	8.05631186376822	8.05631186376822\\
2.5	14	9.06726883094124	9.06726883094124\\
2.5	15	10.0782257981143	10.0782257981143\\
2.5	16	11.0891827652873	11.0891827652873\\
2.75	1	-5.09440384952328	-5.09440384952328\\
2.75	2	-4.08344688235026	-4.08344688235026\\
2.75	3	-3.07248991517725	-3.07248991517725\\
2.75	4	-2.06153294800423	-2.06153294800423\\
2.75	5	-1.05057598083121	-1.05057598083121\\
2.75	6	-0.0396190136581911	-0.0396190136581911\\
2.75	7	0.971337953514827	0.971337953514827\\
2.75	8	1.98229492068785	1.98229492068785\\
2.75	9	2.99325188786086	2.99325188786086\\
2.75	10	4.00420885503388	4.00420885503388\\
2.75	11	5.0151658222069	5.0151658222069\\
2.75	12	6.02612278937992	6.02612278937992\\
2.75	13	7.03707975655294	7.03707975655294\\
2.75	14	8.04803672372596	8.04803672372596\\
2.75	15	9.05899369089897	9.05899369089897\\
2.75	16	10.069950658072	10.069950658072\\
3	1	-6.11363595673857	-6.11363595673857\\
3	2	-5.10267898956555	-5.10267898956555\\
3	3	-4.09172202239253	-4.09172202239253\\
3	4	-3.08076505521951	-3.08076505521951\\
3	5	-2.06980808804649	-2.06980808804649\\
3	6	-1.05885112087347	-1.05885112087347\\
3	7	-0.0478941537004554	-0.0478941537004554\\
3	8	0.963062813472563	0.963062813472563\\
3	9	1.97401978064558	1.97401978064558\\
3	10	2.9849767478186	2.9849767478186\\
3	11	3.99593371499162	3.99593371499162\\
3	12	5.00689068216464	5.00689068216464\\
3	13	6.01784764933766	6.01784764933766\\
3	14	7.02880461651067	7.02880461651067\\
3	15	8.03976158368369	8.03976158368369\\
3	16	9.05071855085671	9.05071855085671\\
3.25	1	-7.13286806395385	-7.13286806395385\\
3.25	2	-6.12191109678083	-6.12191109678083\\
3.25	3	-5.11095412960781	-5.11095412960781\\
3.25	4	-4.09999716243479	-4.09999716243479\\
3.25	5	-3.08904019526177	-3.08904019526177\\
3.25	6	-2.07808322808875	-2.07808322808875\\
3.25	7	-1.06712626091574	-1.06712626091574\\
3.25	8	-0.0561692937427178	-0.0561692937427178\\
3.25	9	0.954787673430301	0.954787673430301\\
3.25	10	1.96574464060332	1.96574464060332\\
3.25	11	2.97670160777634	2.97670160777634\\
3.25	12	3.98765857494936	3.98765857494936\\
3.25	13	4.99861554212237	4.99861554212237\\
3.25	14	6.00957250929539	6.00957250929539\\
3.25	15	7.02052947646841	7.02052947646841\\
3.25	16	8.03148644364143	8.03148644364143\\
3.5	1	-8.15210017116913	-8.15210017116913\\
3.5	2	-7.14114320399611	-7.14114320399611\\
3.5	3	-6.13018623682309	-6.13018623682309\\
3.5	4	-5.11922926965007	-5.11922926965007\\
3.5	5	-4.10827230247705	-4.10827230247705\\
3.5	6	-3.09731533530404	-3.09731533530404\\
3.5	7	-2.08635836813102	-2.08635836813102\\
3.5	8	-1.075401400958	-1.075401400958\\
3.5	9	-0.0644444337849794	-0.0644444337849794\\
3.5	10	0.946512533388038	0.946512533388038\\
3.5	11	1.95746950056106	1.95746950056106\\
3.5	12	2.96842646773407	2.96842646773407\\
3.5	13	3.97938343490709	3.97938343490709\\
3.5	14	4.99034040208011	4.99034040208011\\
3.5	15	6.00129736925313	6.00129736925313\\
3.5	16	7.01225433642615	7.01225433642615\\
3.75	1	-9.17133227838441	-9.17133227838441\\
3.75	2	-8.16037531121139	-8.16037531121139\\
3.75	3	-7.14941834403837	-7.14941834403837\\
3.75	4	-6.13846137686535	-6.13846137686535\\
3.75	5	-5.12750440969233	-5.12750440969233\\
3.75	6	-4.11654744251932	-4.11654744251932\\
3.75	7	-3.1055904753463	-3.1055904753463\\
3.75	8	-2.09463350817328	-2.09463350817328\\
3.75	9	-1.08367654100026	-1.08367654100026\\
3.75	10	-0.0727195738272428	-0.0727195738272428\\
3.75	11	0.938237393345775	0.938237393345775\\
3.75	12	1.94919436051879	1.94919436051879\\
3.75	13	2.96015132769181	2.96015132769181\\
3.75	14	3.97110829486483	3.97110829486483\\
3.75	15	4.98206526203785	4.98206526203785\\
3.75	16	5.99302222921087	5.99302222921087\\
4	1	-10.1905643855997	-10.1905643855997\\
4	2	-9.17960741842667	-9.17960741842667\\
4	3	-8.16865045125365	-8.16865045125365\\
4	4	-7.15769348408063	-7.15769348408063\\
4	5	-6.14673651690762	-6.14673651690762\\
4	6	-5.1357795497346	-5.1357795497346\\
4	7	-4.12482258256158	-4.12482258256158\\
4	8	-3.11386561538856	-3.11386561538856\\
4	9	-2.10290864821554	-2.10290864821554\\
4	10	-1.09195168104252	-1.09195168104252\\
4	11	-0.0809947138695062	-0.0809947138695062\\
4	12	0.929962253303513	0.929962253303513\\
4	13	1.94091922047653	1.94091922047653\\
4	14	2.95187618764955	2.95187618764955\\
4	15	3.96283315482257	3.96283315482257\\
4	16	4.97379012199559	4.97379012199559\\
};
\end{axis}

\end{tikzpicture}%
    \caption{Plotting \(y\) against \(x\) (left) and the derived feature \(z = \phi(x)\) (middle). The 3D plot (right) graphs the output \(y\) against \(x\) and \(x^2\)}
    \label{fig:feature-map}
\end{figure}
Using polynomial regression, we fit the model \(y \sim 1 + x + x^2\).
% See \Cref{fig:feature-map}.
We can derive a new variable from \(x\) to get \[z = \phi(x) = a_0 + a_1 x + a_2 x^2.\]
By transforming our original data, the quadradic relationship between \(x\) and \(y\) can be viewed as a linear relationship between \(z\) and \(y\).
This shows that polynomial regression is just a special case of multiple regression where each power of \(x\) is treated as a separate dimension.
See \Cref{fig:feature-map}

Here, we make the distinction between different kinds of input variables.
We say that \(x\) is an \textbf{attribute} and that \(z\) is a \textbf{feature}.
% \end{example}
\section{Principal Component Analysis}
\label{sec:principal-component-analysis}

When analyzing data, it can be convenient to transform the given input variables to produce new features.
For a well-chosen transform, these features may be approximated using fewer dimensions than the original input space \cite{koutroumbas2008pattern}.
This is an example of a data preprocessing technique known as \textit{dimension reduction} and can reveal low-dimensional structure.

Principal component analysis (PCA) is an orthogonal coordinate transform that is suitable for dimension reduction if some of the inputs are linearly correlated.
In this case, PCA transforms redundant variables in the input space producing uncorrelated variables in the feature space.

There are a number of ways to derive the optimal PCA transform.
One approach presented in \cite{koutroumbas2008pattern} is based on finding uncorrelated features.
It is straightforward to show that uncorrelated features have a diagonal covariance matrix.
This can be used to solve for the covariance matrix \(C\) of input variables.
By asserting the orthogonality of the PCA transform, we obtain \(V\) from the diagonalization of the covariance matrix \(C = VDV^\top\).
Given this PCA transform, we can show that \(V\) minimizes projection residuals as in \cite{shalizi2021advanced}.

\begin{remark}
    \label{rmk:inner-outer-product-notation}
    The following derivations make use of both inner products and outer products.
    If \(x\) is a column vector, then \(x^\top x\) represents an inner product\footnote{
        We will define inner product more precisely in \Cref{def:inner-product}.
        Until then, \(\ipt{\cdot, \cdot}\) will only be used as a dot product of vectors.
    } and the result is a scalar.
    But, if \(x\) is a row vector, then \(x^\top x\) represents an outer product and the result is a matrix.
    To avoid this ambiguity, we will use the notations \(\ipt{\cdot, \cdot}\) and \(\otimes\) to indicate inner product and outer product, respectively.

    Let \(x = [x_i], y = [y_i] \in \RR^n\) and define
    \begin{equation}
        \label{eqn:inner-product-notation}
        \ipt{x,y} = \sum_{i=1}^{n} x_i y_i \in \RR
    \end{equation}
    and
    \begin{equation}
        \def\medmat#1{\scalebox{.9}{\(\begin{bmatrix} #1 \end{bmatrix}\)}}
        \label{eqn:outer-product-notation}
        x \otimes y
        = [x_i y_j]_{ij}^{n \times n}
        = \medmat{
            x_1 y_1 & x_1 y_2 & \cdots & x_1 y_n\\
            x_2 y_1 & x_2 y_2 & \cdots & x_2 y_n\\
            \vdots & \vdots & \ddots & \vdots \\
            x_n y_1 & x_n y_2 & \cdots & x_n y_n
        } \in \RR^{n \times n}.
    \end{equation}
\end{remark}


\subsection{Finding uncorrelated features}
\label{sub:finding-uncorrelated-features}
The correlation between two random variables \(x\) and \(y\) is defined as
\begin{equation}
    \corr(x,y) = \frac{E[(x-\mu_x)(y-\mu_y)]}{\sigma_x \sigma_y},
\end{equation}
where \(\mu_x\), \(\mu_y\) and \(\sigma_x\), \(\sigma_y\) are the respective means and standard deviations of \(x\) and \(y\).
% If \(\mu_x = \mu_y = 0\), then we say \(x\) and \(y\) are \textit{centered}.
We say \(x\) and \(y\) are uncorrelated when \(\corr(x,y) = 0\).
This happens if and only if
\begin{equation}
    \label{eqn:covariance-formula}
    E[(x-\mu_x)(y-\mu_y)] = \cov(x,y) = 0.
\end{equation}
The covariance matrix for a multivariate random variable \(x = [x_1, x_2, \dots, x_d]\) (as a row vector) has \(\cov(x_i, x_j)\) in the \(i\)-th row and \(j\)-th column.
Then
\begin{equation}
    \label{eqn:covariance-matrix-formula}
    E[(x-\mu_x)^\top (x-\mu_x)] =
    \begin{bmatrix}
        \cov(x_1, x_1) & \cov(x_1, x_2) & \cdots & \cov(x_1, x_d) \\
        \cov(x_2, x_1) & \cov(x_2, x_2) & \cdots & \cov(x_2, x_d) \\
        \vdots         & \vdots         & \ddots & \vdots         \\
        \cov(x_d, x_1) & \cov(x_d, x_2) & \cdots & \cov(x_d, x_d) \\
    \end{bmatrix}.
\end{equation}
% If each \(x_i\) is centered, then left hand side of \cref{eqn:covariance-matrix-formula} becomes \(E(x^\top x)\).
If \(x_1, x_2, \dots, x_d\) are pairwise uncorrelated, then \(\cov(x_i, x_j) = 0\) for all \(i \neq j\).
Hence, uncorrelated variables have a diagonal covariance matrix.

Now, let \(a_1, a_2, \dots, a_n \in \RR^{1 \times d}\) represent \(n\) observations in \(d\) variables.
These observations can be considered points in \(d\)-dimensional space whose centroid is \(\mu_a = \frac{1}{n} \sum_{i=1}^{n} a_i\).
We want to determine a PCA transform which sends these points in the input space to points in the feature space.
Moreover, the basis vectors of the feature space shall be uncorrelated.
Accordingly, let \(V \in \RR^{d \times d}\) be the change of basis matrix and let
\begin{equation}
    \label{eqn:transformed-points}
    b_i = (a_i - \mu_a) V, \quad \text{for \(i = 1, 2, \dots, n\)}
\end{equation}
be observations with respect to the feature coordinates.
Then
\begin{equation}
    \mu_b
    = \frac{1}{n} \sum_{i=1}^{n} b_i
    = \frac{1}{n} \sum_{i=1}^{n} (a_i - \mu_a) V
    = 0.
\end{equation}
Using \cref{eqn:covariance-matrix-formula}, we can compute the sample covariance matrices as 
\begin{equation}
    \label{eqn:covariance-matrix-pca}
    C = \frac{1}{n-1} \sum_{i=1}^{n} (a_i - \mu_a)^\top (a_i - \mu_a),
    \qquad
    D = \frac{1}{n-1} \sum_{i=1}^{n} b_i^\top b_i.
\end{equation}
Since \(D\) is the covariance matrix of uncorrelated features, by the argument above, it is diagonal.
If we restrict \(V\) to be orthogonal, then
\begin{equation}
    \label{eqn:diagonalize-covariance}
    b_i^\top b_i = V^\top (a_i - \mu_a)^\top (a_i - \mu_a) V
    \implies
    D = V^\top C V
    \implies
    C = V D V^\top.
\end{equation}
Hence, \(V\) must be a matrix of orthonormal eigenvectors \(v_1, v_2, \dots, v_d\) corresponding to eigenvalues \(\lambda_1, \lambda_2, \dots, \lambda_d\) on the diagonal of \(D\).
When the eigenvalues and eigenvectors are ordered such that
\begin{equation}
    \lambda_1 \geq \lambda_2 \geq \cdots \geq \lambda_d,
\end{equation}
we call \(v_1, v_2, \dots, v_d\) the \textit{principal components} of the PCA transform matrix \(V\).



\subsection{Singular value decomposition}
\label{sub:singular-value-decomposition}
% TODO: SVD
The singular value decomposition (SVD) of a rectangular matrix generalizes the idea of diagonalization for square matrices.
Moreover, this provides a connection between the matrices \(A^\top A\) and \(AA^\top\).

\begin{theorem}[Singular value decomposition]
    \label{thm:svd}
    \cite{horn2013matrix}
    Let \(A \in \RR^{n \times d}\).
    Then there exist orthogonal matrices \(U \in \RR^{n \times n}\) and \(V \in \RR^{d \times d}\) and a diagonal matrix \(S \in \RR^{n \times d}\) such that \(A = USV^\top\).
    We say the columns of \(U = \begin{bmatrix}
        u_1 & u_2 & \cdots & u_n
    \end{bmatrix}\) and \(V = \begin{bmatrix}
        v_1 & v_2 & \cdots & v_d
    \end{bmatrix}\) are the left and right singular vectors of \(A\), respectively
    The diagonal entries of \(S\) are the called the singular values \(\sigma_1, \sigma_2, \dots, \sigma_r\), where \(r = \rank A \leq \min\{n,d\}\).
    Then we can write
    \begin{equation}
        \label{eqn:svd}
        A = USV^\top = \sum_{i=1}^{r} \sigma_i u_i v_i^\top.
    \end{equation}
\end{theorem}

The SVD of a matrix \(A\) can be found by diagonalizing \(A^\top A\) and \(AA^\top\).
If \(A = USV^\top\), then
\begin{align*}
    A^\top A &= (USV^\top)^\top (USV^\top) = V S U^\top U S V^\top = V S^2 V^\top\\
    AA^\top  &= (USV^\top) (USV^\top)^\top = U S V^\top V S U^\top = U S^2 U^\top.
\end{align*}

So, \(\{v_j\}_{j=1}^d\) are the eigenvectors of \(A^\top A\), \(\{u_j\}_{j=1}^n\) are the eigenvectors of \(AA^\top\), and \(\{\sigma_j^2\}_{j=1}^r\) are the eigenvalues of both \(A^\top A\) and \(AA^\top\).
Notice that the SVD of \(A\) will give us the projection matrix \(V\) in \cref{eqn:diagonalize-covariance}, provided that \(A\) is centered.
In this way, we see that PCA is really just a special case of the SVD.

\begin{theorem}[Frobenius norm]
    \label{def:frobenius}
    \cite{mohri2012foundations,horn2013matrix}
    The \textit{Frobenius norm} (or \textit{Hilbert-Schmidt norm}) of a matrix \(A = [a_{ij}] \in \RR^{n \times d}\) is given by
    \begin{equation}
        \label{eqn:frobenius}
        \|A\|_F = \sqrt{\tr(A^\top A)} = \sqrt{\sum_{i=1}^{n} \sum_{j=1}^{d} a_{ij}^2}.
    \end{equation}
\end{theorem}
\begin{proof}
    Let \(A = [a_{ij}] \in \RR^{n \times d}\).
    Then \(A^\top A = \left[\sum_{k=1}^{n} a_{k i} a_{k j}\right]_{ij}\).
    It follows that \(\|A\|_F^2 = \tr(A^\top A) = \sum_{i=1}^{n} \sum_{j=1}^{d} a_{ij}^2\).
    Clearly, \(\|A\|_F > 0\) whenever \(A\) is not the zero matrix and \(\|A\|_F = 0\) whenever \(A\) is the zero matrix.

    For the triangle inequality, consider another matrix \(B = [b_{ij}] \in \RR^{d \times m}\).
    Then
    \begin{align*}
        \|AB\|_F
        &= \sqrt{\sum_{i=1}^{n} \sum_{j=1}^{m} \sum_{k=1}^{d} (a_{ik} b_{kj})^2}\\
        &\leq \sqrt{\sum_{i=1}^{n} \sum_{j=1}^{m}
        \left(\sum_{k=1}^{d} a_{ik}^2\right) \left(\sum_{k=1}^{d} b_{kj}^2\right)}\\
        &= \sqrt{\sum_{i=1}^{n} \sum_{j=1}^{d} a_{ij}^2}
        \sqrt{\sum_{i=1}^{d} \sum_{j=1}^{m} b_{ij}^2}\\
        &= \|A\|_F \|B\|_F.
    \end{align*}
    Thus, \(\|\cdot\|_F\) is a matrix norm.
\end{proof}

Combining \cref{eqn:svd,eqn:frobenius}, we have
\begin{equation}
    \label{eqn:frobenius-singular-values}
    \|A\|_F = \sqrt{\sum_{i=1}^{r} \sigma_i^2} = \sqrt{\sum_{i=1}^{r} \lambda_i^2},
\end{equation}
where \(\{\sigma_i\}_{i=1}^r\) are the singular values of \(A\) and \(\{\lambda_i\}_{i=1}^r\) are the eigenvalues of \(A^\top A\) or \(AA^\top\).

\subsection{Minimizing projection residuals}
\label{sub:minimizing-projection-residuals}
In this section, we want to show that the PCA projection minimizes the residual error.

Suppose \(a_1, a_2, \dots, a_n \in \RR^n\).
Let \(v_1, v_2, \dots, v_d \in \RR^d\) be the principal component vectors given in \Cref{sub:finding-uncorrelated-features}.
Define the projections onto the subspace spanned by \(v_1, v_2, \dots, v_p\), for \(p \leq d\), as
\begin{equation}
    \hat{a}_i = \sum_{j=1}^{p} \ipt{a_i, v_j} v_j,
    \quad \text{for \(i = 1,2,\dots, n\)}.
\end{equation}
Then each residual from the projection is given by
\begin{align}
    \label{eqn:projection-residuals}
    \|a_i - \hat{a}_i\|^2
    &= \left\langle a_i - \hat{a}_i, a_i - \hat{a}_i\right\rangle
    \\
    % &= \left\langle a_i, a_i \right\rangle
    % - 2\left\langle a_i, \hat{a}_i \right\rangle
    % + \left\langle \hat{a}_i, \hat{a}_i \right\rangle
    % \notag \\
    &= \left\| a_i \right\|^2
    - 2\left\langle a_i, \hat{a}_i \right\rangle
    + \left\| \hat{a}_i \right\|^2
    \notag \\
    &= \left\| a_i \right\|^2
    - 2\left\langle a_i, \sum_{j=1}^{p} \langle a_i, v_j \rangle v_j \right\rangle
    + \left\| \sum_{j=1}^{p} \langle a_i, v_j \rangle v_j \right\|^2
    \notag \\
    &= \left\| a_i \right\|^2
    - 2\sum_{j=1}^{p} \langle a_i, v_j \rangle \left\langle a_i,  v_j \right\rangle
    + \sum_{j=1}^{p} \langle a_i, v_j \rangle^2 \left\|  v_j \right\|^2 
    \notag \\
    &= \left\| a_i \right\|^2
    - 2\sum_{j=1}^{p} \langle a_i, v_j \rangle^2
    + \sum_{j=1}^{p} \langle a_i, v_j \rangle^2
    \notag \\
    &= \left\| a_i \right\|^2
    - \sum_{j=1}^{p} \langle a_i, v_j \rangle^2.
    \notag
\end{align}
Then the mean squared error to be minimized is
\begin{equation}
    \MSE =
    \frac{1}{n} \sum_{i=1}^{n} \|a_i\|^2 - \frac{1}{n} \sum_{i=1}^{n} \sum_{j=1}^{p} \langle a_i, v_j \rangle^2.
\end{equation}
Since the first term does not depend on \(v_j\), the \(\MSE\) is minimized whenever
% \begin{equation}
    \(\sum_{i=1}^{n} \sum_{j=1}^{p} \ipt{a_i, v_j}^2\)
% \end{equation}
is maximized.
Let \(A\) be the matrix whose rows are \(a_1, a_2, \dots, a_n\) and \(V_p\) be the matrix whose columns are \(v_1, v_2, \dots, v_p\).
By \Cref{thm:frobenius,eqn:frobenius-singular-values}, this becomes
\begin{equation}
    \sum_{i=1}^{n} \sum_{j=1}^{p} \ipt{a_i, v_j}^2
    = \tr[(AV_p)^\top AV]
    = \|AV_p\|_F
    = \sum_{i=j}^{p} \lambda_j.
\end{equation}
It follows that the vectors \(v_1, v_2, \dots, v_p\) which minimize projection error correspond to the \(p\) largest eigenvalues given by \cref{eqn:eigenvalue-order}.

\subsection{PCA algorithm}
Let \(A\) be a data matrix whose \(n\) rows correspond to observations and \(d\) columns correspond to variables.
The following algorithm demonstrates a simple method for computing the PCA of \(A\):
\begin{enumerate}
    \item Compute the centered matrix \(A_0 = A - \colmean(A)\).
    \item Compute the covariance matrix \(C = \frac{1}{n-1} A_0^\top A_0\).
    \item Diagonalize the covariance matrix such that \(C = V D V^\top\).
    \item Order the eigenvalues and eigenvectors so that \(\lambda_1 \geq \lambda_2 \geq \cdots \geq \lambda_d\).
    We call the ordered eigenvalues the \textit{principal components}.
    \item Choose the dimension of the subspace \(p \leq d\).
    \item Construct the \(d \times p\) projection matrix \(V_p\) using the first \(p\) principal components \(v_1, v_2, \dots, v_p\).
\end{enumerate}

\begin{example}
    \label{eg:pca}
    \def\Amat{\begin{bmatrix}
    5 &  3 &  6 &  7 &  6 \\
    4 &  5 &  7 &  1 &  3 \\
    5 &  7 &  6 &  1 &  0 \\
    6 & 10 & 12 & 12 & 11 \\
    9 & 10 & 12 & 13 &  9 \\
\end{bmatrix}}
Consider the following matrix
\begin{equation}
    A = \Amat.
\end{equation}
The column means are \(\mu = [5.8,7,8.6,6.8,5.8]\).
Then the mean-centered data becomes
\def\-{\phantom{-}}
\begin{equation}
    A_0 = A - \mu = \frac{1}{5}\begin{bmatrix}
        -4 &   -20 &   -13 &   \-1 &   \-1 \\
        -9 &   -10 &    -8 &   -29 &   -14 \\
        -4 &   \-0 &   -13 &   -29 &   -29 \\
        \-1 &  \-15 &  \-17 &  \-26 &  \-26 \\
    \-16 &  \-15 &  \-17 &  \-31 &  \-16 \\
\end{bmatrix}.
\end{equation}

The covariance matrix is
\begin{equation}C = A_0^\top A_0 = \frac{1}{5}\begin{bmatrix}
        74 &  85 &  93 & 179 & 104 \\
        85 & 190 & 170 & 225 & 150 \\
        93 & 170 & 196 & 313 & 238 \\
    179 & 225 & 313 & 664 & 484 \\
    104 & 150 & 238 & 484 & 394 \\
\end{bmatrix}.\end{equation}

Diagonalizing \(C\) gives
\begin{align*}
    V &= \begin{bmatrix}
        0.1888 & -0.2020 & -0.6366 &  0.5495 & -0.4651\\
        0.2755 & -0.7886 &  0.1472 & -0.4502 & -0.2791\\
        0.3606 & -0.3464 &  0.3128 &  0.5836 &  0.5582\\
        0.6979 &  0.2522 & -0.4422 & -0.3707 &  0.3411\\
        0.5209 &  0.3922 &  0.5288 &  0.1316 & -0.5271\\
    \end{bmatrix},\\
    D &= \begin{bmatrix}
        264.8458 &       0 &      0 &      0 & 0 \\
                0 & 27.9766 &      0 &      0 & 0 \\
                0 &       0 & 9.3198 &      0 & 0 \\
                0 &       0 &      0 & 1.4579 & 0 \\
                0 &       0 &      0 &      0 & 0 \\
    \end{bmatrix}.
\end{align*}

If we keep all \(5\) principal component vectors, then \(V_5 = V\) and the projection of \(A_0\) along \(V\) is
\begin{equation}B = A_0 V = \begin{bmatrix}
    -1.9469 &  4.3453 & -0.8756 & -0.2039 & 0 \\
    -6.9742 & -0.0660 &  1.4352 &  0.7590 & 0 \\
    -8.1577 & -2.6752 & -0.8063 & -0.5704 & 0 \\
        8.4282 & -0.2330 &  1.8282 & -0.4996 & 0 \\
        8.6507 & -1.3711 & -1.5815 &  0.5149 & 0 \\
\end{bmatrix}.\end{equation}
Here, the last column of \(B\) is the zero vector because the last eigenvalue of \(C\) is zero\footnote{Since we subtracted the column means from a square matrix \(A\), the dimension of the row space was reduced to 4.}.
To perfectly reconstruct \(A\), we need \(k = 4\) principal components and the row vector \(\mu\)
\begin{equation}A = B V^\top + \mu = B V_4^\top + \mu.\end{equation}

If we use \(k = 3\) principal components, then the projection of \(A_0\) onto \(V_3\) is
\begin{equation}B = A_0 V_3 = \begin{bmatrix}
    -1.9469 &  4.3453 & -0.8756 \\
    -6.9742 & -0.0660 &  1.4352 \\
    -8.1577 & -2.6752 & -0.8063 \\
        8.4282 & -0.2330 &  1.8282 \\
        8.6507 & -1.3711 & -1.5815 \\
\end{bmatrix}\end{equation}
and \(A\) is approximately reconstructed by
\begin{equation}A \approx B V_3^\top + \mu = \begin{bmatrix}
    % 5.1121 &  2.9082 &  6.1190 &  6.9244 &  6.0268 \\
    % 3.5829 &  5.3417 &  6.5570 &  1.2814 &  2.9001 \\
    % 5.3134 &  6.7432 &  6.3329 &  0.7886 &  0.0751 \\
    % 6.2745 &  9.7751 & 12.2916 & 11.8148 & 11.0658 \\
    % 8.7170 & 10.2318 & 11.6995 & 13.1909 &  8.9322 \\
    5.1 &  2.9 &  6.1 &  6.9 &  6.0 \\
    3.6 &  5.3 &  6.6 &  1.3 &  2.9 \\
    5.3 &  6.7 &  6.3 &  0.8 &  0.1 \\
    6.3 &  9.8 & 12.3 & 11.8 & 11.1 \\
    8.7 & 10.2 & 11.7 & 13.2 &  8.9 \\
\end{bmatrix}.\end{equation}
We can compute the reconstruction error using the Frobenius norm
\begin{equation*}%\label{eqn:reconstruction_error}
    E_k = \|A - (V_k^\top + \mu)\|_F.
\end{equation*}
By the SVD, we have \(A_0 = USV^\top\), where \(S = \sqrt{D}\).
So, the projection of \(A_0\) onto \(V_p\) is \begin{equation}B = A_0 V_p = U S_p,\end{equation} where \(S_p\) is the diagonal matrix of the first \(p\) singular values.
Then the reconstruction error becomes
\begin{align*}
    \|A - (BV_p^\top + \mu)\|_F
    &= \|(A - \mu) - BV_p^\top\|_F \\
    &= \|A_0 - BV_p^\top\|_F \\
    &= \|USV^\top - US_pV^\top\|_F \\
    &= \|U(S - S_p)V^\top\|_F \\
    &= \|S-S_p\|_F \\
    &= \sqrt{\sigma_{p+1}^2 + \dots + \sigma_{d}^2}.
\end{align*}
Hence,
\begin{equation}E_3 = \sigma_3 + \sigma_4 = \sqrt{1.4579^2 + 0^2} = 1.2074.\end{equation}
\end{example}

\subsection{Applications of PCA}

One of the most common applications of PCA is \textit{dimension reduction}.
When variables are correlated, the observations lie in some linear subspace of the original space.
In this situation, PCA can be used to project down to the lower dimension and have the smallest possible projection error.
This is particularly useful when the dimension of the input space is extremely large.
A number challenges arise when analyzing high-dimensional data and are collectively referred to as the \textit{curse of dimensionality} \cite{koutroumbas2008pattern}.
By working in the PCA feature space, these problems may be avoided.

\begin{example}
    \label{eg:pca-code}
    \begin{figure*}
    \centering
    \makebox[\textwidth][c]{%
    \begin{minipage}{6.5in}
        \centering
        \includegraphics[width=.49\textwidth]{figs/fig-pca-code-uncorr.pdf}
        \includegraphics[width=.49\textwidth]{figs/fig-pca-code-corr.pdf}
    \end{minipage}}
    \caption{PCA projection of three-dimensional data onto two-dimensional subspace.}
    \label{fig:pca-code}
\end{figure*}
In the following experiment, we consider two sets of three-dimensional data.
Each set of vectors \(x_1, x_2, x_3 \in \RR^n\) are sampled from a standard normal distribution with sample size \(n = 30\).
Suppose in the first set there is no apparent relationship among variables while, in second set, we have \(x_1 = 4x_2 + 2x_3\).
For simplicity, we say the first set is \textit{uncorrelated} while the second set is \textit{correlated}.
See \Cref{fig:pca-code}.
In the uncorrelated data, the reconstruction error is \(~ 3.92\), which does not seem to indicate the presence of a pattern.
The reconstruction error for the correlated data is \(~ 0.973\).
This error can be explained by the variation of \(x_2\) with \(x_3\).
Meanwhile, the correlated pair plots indicate a pattern among \(x_2\) vs \(x_1\) and \(x_3\) vs \(x_1\).
\end{example}

% \subsection{Linear regression and PCA}
% \def\vb#1{\mathbf{#1}}
\cite{shawe2004kernel}
Let \(\vb{x}_1, \dots, \vb{x}_m \in \RR^n\) be observation vectors and \(\vb{y} \in \RR^n\) be a target vector.
A linear regression model finds a vector of weights \(\vb{w} = (w_1, w_2, \dots, w_n)\) to determine the linear function
\begin{equation}
    f(\vb{x}) = \vb{w} \cdot \vb{x} = \sum_{i=1}^{n} w_i x_i,
\end{equation}
where \(\vb{x} = (x_1, x_2, \dots, x_n)\) is an input vector.
The residual error of each observation is \(\vb{y} - f(\vb{x}_i)\), for \(i = 1,2, \dots, m\).
Then the residual sum of squares is given to be
\begin{equation}
    \label{eqn:residual-sum-of-squares}
    RSS
    = \sum_{i=1}^{m} (\vb{y} - f(\vb{x}_i))^2
    = \sum_{i=1}^{m} (\vb{y} - \vb{w} \cdot \vb{x}_i)^2
    = (\vb{y} - X^\top \vb{w})^\top (\vb{y} - X^\top \vb{w}),
\end{equation}
where \(X = [\vb{x}_1, \dots, \vb{x}_m]\).
By minimizing \(RSS\), the magnitude of the residuals will be as small as possible, producing the optimal linear model \(f(\vb{x})\).
Therefore, this method is known as least squares regression.
Differentiate \eqref{eqn:residual-sum-of-squares} with respect to \(\vb{w}\) and set equal to zero so that
\begin{equation}
    2 X^\top \vb{y} - 2 X^\top X \vb{w} = 0.
\end{equation}
This leads to the normal equation \(X^\top X \vb{w} = X^\top \vb{y}\).
Thus,
\begin{equation}
    \vb{w} = (X^\top X)^{-1} X^\top \vb{y}
\end{equation}
gives the optimal linear model which minimizes residual error.

% \begin{example}
%     % TODO: Decide on PCA regression

In two dimensions, let \(\vb{x} = (x_1, \dots, x_m)\) and \(\vb{y} = (y_1, \dots, y_m)\).
Then the least squares regression model is given by
\begin{equation}
\label{eqn:ols-model}
f(x) = \frac{\cov(\vb{x},\vb{y})}{\Var(\vb{x})}(x - \overline{\vb{x}}) + \overline{\vb{y}},
\end{equation}
Since \(\vb{x}\) is treated as an input variable and \(\vb{y}\) is treated as an output, this is a type of supervised learning.
In contrast, PCA is an unsupervised learning technique.
PCA organizes the variables in the input space to reveal any patterns in the underlying data.
% \footnote{I think this model would be considered total least squares (TLS) regression since it uses all variables (input and output). Principal component regression (PCR) creates the regression equation using only the input variables.}
If we have input variables \(\vb{x}_1\) and \(\vb{x}_2\), we can use PCA to find a linear pattern among the inputs without trying to predict an output.
First, we find the direction of the largest variance \(\lambda_{\max}\) using the covariance matrix
\[\begin{bmatrix}
\cov(\vb{x}_1, \vb{x}_1) & \cov(\vb{x}_1, \vb{x}_2) \\
\cov(\vb{x}_2, \vb{x}_1) & \cov(\vb{x}_2, \vb{x}_2)
\end{bmatrix} = \begin{bmatrix}
\Var(\vb{x}_1) & \cov(\vb{x}_1, \vb{x}_2) \\
\cov(\vb{x}_1, \vb{x}_2) & \Var(\vb{x}_2)
\end{bmatrix}.\]
In particular,
\[\lambda_{\max} = \frac{1}{2}\left(\Var(\vb{x}) + \Var(\vb{y}) + \sqrt{(\Var(\vb{x}) - \Var(\vb{y}))^2 + 4\cov(\vb{x},\vb{y})^2}\right).\]
Then the PCA regression model becomes
\begin{equation}
\label{eqn:pca-model}
g(x) = \frac{\lambda_{\max} - \Var({\vb{x}_1})}{\cov(\vb{x}_1,\vb{x}_2)}(x - \overline{\vb{x}}_1) + \overline{\vb{x}}_2.
\end{equation}
In this case, the projection residuals are orthogonal to the PCA regression line.
See \Cref{fig:ols-pca}.

\begin{figure}
\centering
\includegraphics[width=.49\linewidth]{figs/fig-least-squares.pdf}
\hfill
\includegraphics[width=.49\linewidth]{figs/fig-pca-fit.pdf}
\caption{Least squares regression model \(f(x)\) (left) minimizes the sum of squared errors while total least squares, i.e., the PCA model \(g(x)\) (right) minimizes the orthogonal projections.}
\label{fig:ols-pca}
\end{figure}
% \end{example}
\section{Reproducing Kernel Hilbert Space}
\label{sec:reproducing-kernel-hilbert-space}
% Recall that the PCA algorithm involves minimizing terms involving the dot product \(x^\top x\).
% We would like to apply a feature map to our variables and replace this dot product with an inner product in the feature space.
% However, since the dimension of the feature space will typically be much larger than the dimension of the original space, optimization using this inner product will be expensive.
% This problem is solved by the so-called \textit{kernel trick} which allows us to circumvent the computational complexity issue caused by high-dimensional spaces.
% In order to justify the kernel trick, we need to establish some theory of reproducing kernel Hilbert spaces.

% First, we recall the definition of an inner product.

In this section, our goal is to establish properties of Hilbert spaces and kernel functions that can be used to modify the PCA algorithm.
To begin, we will briefly cite some definitions and results from analysis \cite{kreyszig1991introductory}, \cite{rudin1987real} and matrix theory \cite{horn2013matrix}.

\begin{definition}
    \cite{small1994hilbert} % page 10
    Let \(X\) be a (real) vector space.
    An \textit{inner product} is a function \(\langle \cdot, \cdot \rangle : X \times X \to \RR\) which satisfies the following properties:
    \begin{enumerate}
        \item Symmetry. For all \(x,y \in X\),
        \[\langle x,y \rangle = \langle y, x \rangle.\]
        \item Linear in the first argument. For all \(x,y,z \in X\), \(\alpha, \beta \in \RR\),
        \[\langle \alpha x + \beta y, z \rangle = \alpha \langle x, z \rangle + \beta \langle y, z \rangle.\]
        \item Positive definite. For all \(x \in X\),
        \[\langle x, x \rangle \geq 0\]
        and \(\langle x, x \rangle = 0\) if and only if \(x = 0\).
    \end{enumerate}
    An \textit{inner product space} is a vector space along with an inner product.
\end{definition}

Since real inner products are symmetric and linear in the first argument,
\[
    \langle z, \alpha x + \beta y\rangle
    = \langle \alpha x + \beta y, z \rangle
    = \alpha \langle x, z \rangle + \beta \langle y, z \rangle
    = \alpha \langle z, x \rangle + \beta \langle z, y \rangle
.\]
So, we also have linearity in the second argument.

The norm induced by an inner product is defined as
\[\|x\| = \langle x, x \rangle^{1/2}\]
and the metric induced by this norm is
\[d(x,y) = \|x - y\| = \langle x-y, x-y \rangle^{1/2}.\]
It follows that an inner product space is also a normed space and a metric space.
So, the induced norm will have the following properties for all \(x,y \in X\) and \(\alpha \in \RR\):
\begin{enumerate}
    \item Triangle inequality. \(\|x + y \| \leq \|x\| + \|y\|\);
    \item Scalar multiplication. \(\|\alpha x\| = |\alpha| \|x\|\);
    \item Positivity. \(\|x\| \geq 0\) and \(\|x\| = 0\) if and only if \(x = 0\). 
\end{enumerate}

\begin{definition}[Hilbert space]
    A Hilbert space is a complete inner product space.
    For a Hilbert space \(H\), we sometimes denote the inner product as \(\langle \cdot, \cdot \rangle_H\) to avoid ambiguity.
\end{definition}

Two Hilbert spaces \(H\) and \(L\) (over the same field) are said to be \textit{isomorphic} if there is a bijection \(T : H \to L\) such that
\begin{equation}
    \label{eqn:hilbert-space-isomorphism}
    \langle x, y \rangle_H = \langle Tx, Ty \rangle_L,
\end{equation}
for every \(x, y \in H\).
In \cite{kreyszig1991introductory}, Kreyszig shows that two Hilbert spaces are isomorphic if and only if they have the same dimension.
If \(V\) is an inner product space that is not complete, then it can be extended to a Hilbert space by completion.
The completion of an inner product space is denoted as \(\overline{V}\) and is unique up to isomorphism.

A Hilbert space is said to be \textit{separable} if it contains a dense countable subset.
It can be shown \cite{kreyszig1991introductory} that a Hilbert space is separable if and only if it has a countable orthonormal basis.
The following example demonstrates a useful property of separable Hilbert spaces.

\begin{example}
    \cite{rudin1987real} % p 84--85
    The space of square-summable (real) sequences is defined as
    \begin{equation}
        \label{eqn:l2-space}
        \ell^2(A) = \left\{
            x : A \to \RR \;\middle\vert\; \sum_{a \in A} x_a^2 < \infty
        \right\}.
    \end{equation}
    Given the inner product
    \begin{equation}
        \label{eqn:l2-inner-product}
        \langle x, y \rangle = \sum_{a \in A} x_a y_a,
    \end{equation}
    \(\ell^2(A)\) is a Hilbert space.
    Moreover, \(\ell^2(A)\) is separable if and only if \(A\) is countable.
    It follows that the sequence space \(\ell^2 = \ell^2(\NN)\) is the separable Hilbert space of square-summable sequences.
    Due to the Riesz-Fischer theorem, every infinite-dimensional Hilbert space is isomorphic to \(\ell^2\).
\end{example}

\begin{definition}[Gram matrix]
    \cite{horn2013matrix}
    Let \(x_1, x_2, \dots, x_n\) be vectors in an inner product space \(X\).
    The matrix \(G \in \RR^{n \times n}\) is called a \textit{Gram matrix} if \([G]_{ij} = \langle x_i, x_j \rangle\).
\end{definition}

\begin{example}
    \def\v{\mathbf{v}}
    Consider the vectors in \(\RR^3\):
    \begin{align*}
        \v_1 &=
        \begin{bmatrix}
            v_{11} \\ v_{21} \\ v_{31}
        \end{bmatrix},&
        \v_2 &=
        \begin{bmatrix}
            v_{12} \\ v_{22} \\ v_{32}
        \end{bmatrix},&
        \v_3 &=
        \begin{bmatrix}
            v_{13} \\ v_{23} \\ v_{33}
        \end{bmatrix},&
        \v_4 &=
        \begin{bmatrix}
            v_{14} \\ v_{24} \\ v_{34}
        \end{bmatrix}.
    \end{align*}
    The Gram matrix for these vectors is
    \begin{align*}
        G = \begin{bmatrix}
            \v_1^\top \v_1 & \v_1^\top \v_2 & \v_1^\top \v_3 & \v_1^\top \v_4\\[3pt]
            \v_2^\top \v_1 & \v_2^\top \v_2 & \v_2^\top \v_3 & \v_2^\top \v_4\\[3pt]
            \v_3^\top \v_1 & \v_3^\top \v_2 & \v_3^\top \v_3 & \v_3^\top \v_4\\[3pt]
            \v_4^\top \v_1 & \v_4^\top \v_2 & \v_4^\top \v_3 & \v_4^\top \v_4\\
        \end{bmatrix}.
    \end{align*}
    If \(V\) is a matrix whose columns are \(\v_1\), \(\v_2\), \(\v_3\), \(\v_4\), then we can write \(G = V^\top V\).
\end{example}

\begin{definition}[kernel]
    Let \(X\) be a nonempty set and \(k : X \times X \to \RR\).
    Then \(k\) is a (positive semi-definite) \textit{kernel} if
    \begin{enumerate}
        \item \(k\) is symmetric: for all \(x,y \in X\),
        \[k(x,y) = k(y,x);\]
        \item \(k\) is positive semi-definite: if \(x_1, \dots, x_n \in X\) and \(c_1, \dots, c_n \in \RR\), then
        \[\sum_{i=1}^{n}\sum_{j=1}^{n} c_i c_j k(x_i, x_j) \geq 0.\]
        Equivalently, the matrix \(K \in \RR^{n \times n}\) whose entries are \([K]_{ij} = k(x_i, x_j)\) is positive semi-definite, that is, \(\mathbf{c}^\top K \mathbf{c} \geq 0\) for all \(\mathbf{c} \in \RR^n\).
        We call \(K\) a \textit{kernel matrix}.
        Note that since \(k\) is symmetric, so is \(K\).
    \end{enumerate}
\end{definition}

\begin{definition}[feature map]
    \cite{rudin2020notes}
    Let \(F\) be a Hilbert space of functions \(f : X \to \RR\).
    A feature map is a function \(\Phi : X \to F\).
\end{definition}

Our goal is to have kernels written as inner products of feature maps:
\[k(x,y) = \langle \Phi(x), \Phi(y) \rangle_{H_k}.\]
If \(\Phi\) is a known feature map, then \(k\) would surely be symmetric and positive definite since inner products are.
If instead we know \(k\) is a kernel, then we need to show that there is a unique Hilbert space with the desired inner product.

\subsection{Constructing a reproducing kernel Hilbert space}

% \subsubsection*{To do:}
% \begin{itemize}
%     \item move kernel/Gram matrix to its own definition
%     \item Define linear functionals
%     \item Define dual space (maybe?)
%     \item Riesz representation theorem
%     \item Define evaluation functionals
%     \item Define RKHS (continuous evaluation functionals)
%     \item An RKHS defines a unique reproducing kernel (by Riesz representation theorem).
%     \item Mercer's theorem.
%     \item A kernel defines a feature map.
%     \item A kernel defines a unique RKHS (by Mercer/Moore-Aronszajn).
%     \item We finally get \(k(x,y) = \langle \Phi(x), \Phi(y) \rangle_H\).
% \end{itemize}

\subsection{Constructing kernels}

\begin{theorem}
    \cite{rudin2020notes,shawe2004kernel}
    Suppose \(k_1\) and \(k_2\) are kernels over \(X \times X\).
    The following functions kernels.
    \begin{enumerate}
        \item \(k(x,y) = a_1 k_1(x,y) + a_2 k_2(x,y)\) for all \(a_1, a_2 \geq 0\).
        \item \(k(x,y) = k_1(x,y) k_2(x,y)\).
        \item \(k(x,y) = a_0 + a_1 k_1(x,y) + a_2 k_1(x,y)^2 + \cdots + a_n k_1(x,y)^n\) for all \(n \in \NN\) and \(a_0, \dots, a_n \geq 0\).
        \item \(k(x,y) = k_1(h(x),h(y))\) for all \(h : X \to X\).
        \item \(k(x,y) = g(x)g(y)\) for all \(g : X \to \RR\).
        \item \(k(x,y) = \exp(k_1(x,y))\).
    \end{enumerate}
\end{theorem}

\begin{proof}
    Let \(x_1, \dots, x_n \in X\) and \(c_1, \dots, c_n \in \RR\).
    \begin{enumerate}
        \item \label{itm:kernel-linear-combo}
        Let \(k = a_1k_1 + a_2k_2\) for \(a_1, a_2 \geq 0\).
        Since \(k_1\) and \(k_2\) are symmetric,
        \[
            k(x,y)
            = a_1k_1(x,y) + a_2k_2(x,y)
            = a_1k_1(y,x) + a_2k_2(y,x)
            = k(y,x),
        \]
        for all \(x,y \in X\).
        So, \(k\) is symmetric.

        Since \(k_1\) and \(k_2\) are positive semi-definite and \(a_1, a_2 \geq 0\),
        \begin{align*}
            \sum_{i=1}^{n} \sum_{j=1}^{n} c_i c_j k(x_i,x_j)
            &= \sum_{i=1}^{n} \sum_{j=1}^{n} c_i c_j (a_1 k_1(x_i,x_j) + a_2 k_2(x_i,x_j))\\
            &= a_1 \sum_{i=1}^{n} \sum_{j=1}^{n} c_i c_j k_1(x_i,x_j)
            + a_2 \sum_{i=1}^{n} \sum_{j=1}^{n} c_i c_j k_2(x_i,x_j)\\
            &\geq 0.
        \end{align*}
        So, \(k\) is positive semi-definite.
        \item \label{itm:kernel-product}
        Let \(k = k_1 k_2\).
        Define \(K\) so that \([K]_{ij} = k(x_i,x_j) = k_1(x_i, x_j) k_2(x_i, x_j)\).
        Let \(K_1\) and \(K_2\) be the Gram matrices for \(k_1\) and \(k_2\), respectively.
        Then \(K_1, K_2\) have orthonormal eigenvectors and nonnegative eigenvalues such that
        \def\dsum{\displaystyle\sum}
        \begin{align*}
            K_1 &= V LV^\top \\
            &= \begin{bmatrix}
                v_{11} & \cdots & v_{1n}\\
                \vdots & \ddots & \vdots\\
                v_{n1} & \cdots & v_{nn}\\
            \end{bmatrix}
            \begin{bmatrix}
                \lambda_{1} & \cdots & 0\\
                \vdots & \ddots & \vdots\\
                0 & \cdots & \lambda_{n}\\
            \end{bmatrix}
            \begin{bmatrix}
                v_{11} & \cdots & v_{n1}\\
                \vdots & \ddots & \vdots\\
                v_{1n} & \cdots & v_{nn}\\
            \end{bmatrix}\\
            &= \begin{bmatrix}
                \dsum_{j=1}^{n} \lambda_{j} v_{1j} v_{1j} & \cdots & \dsum_{j=1}^{n} \lambda_{j} v_{nj} v_{1j} \\
                \vdots & \ddots & \vdots\\
                \dsum_{j=1}^{n} \lambda_{j} v_{1j} v_{nj} & \cdots & \dsum_{j=1}^{n} \lambda_{j} v_{nj} v_{nj}\\
            \end{bmatrix}\\
            &= \sum_{j=1}^{n} \lambda_{j}
            \begin{bmatrix}
                v_{1j} v_{1j} & \cdots & v_{nj} v_{1j} \\
                \vdots & \ddots & \vdots\\
                v_{1j} v_{nj} & \cdots & v_{nj} v_{nj}\\
            \end{bmatrix}
            \intertext{and}
            K_2 &= UMU^\top = \sum_{j=1}^{n} \mu_{j}
            \begin{bmatrix}
                u_{1j} u_{1j} & \cdots & u_{nj} u_{1j} \\
                \vdots & \ddots & \vdots\\
                u_{1j} u_{nj} & \cdots & u_{nj} u_{nj}\\
            \end{bmatrix}.
        \end{align*}
        \def\v{\mathbf{v}}
        \def\u{\mathbf{u}}
        Let \(\v_i = \begin{bmatrix}
            v_{1i} & \cdots & v_{ni}
        \end{bmatrix}^\top\) and \(\u_j = \begin{bmatrix}
            u_{1j} & \cdots & u_{nj}
        \end{bmatrix}\), for all \(i,j = 1, 2, \dots, n\).
        Then
        \begin{align*}
            K &= K_1 \circ K_2\\
            &= \sum_{i=1}^{n} \lambda_{i}
            \begin{bmatrix}
                v_{1i} v_{1i} & \cdots & v_{ni} v_{1i} \\
                \vdots & \ddots & \vdots\\
                v_{1i} v_{ni} & \cdots & v_{ni} v_{ni}\\
            \end{bmatrix} \circ
            \sum_{j=1}^{n} \mu_{j}
            \begin{bmatrix}
                u_{1j} u_{1j} & \cdots & u_{nj} u_{1j} \\
                \vdots & \ddots & \vdots\\
                u_{1j} u_{nj} & \cdots & u_{nj} u_{nj}\\
            \end{bmatrix}\\
            &= \sum_{i=1}^{n} \sum_{j=1}^{n} \lambda_{i} \mu_{j}
            \begin{bmatrix}
                v_{1i} u_{1j} v_{1i} u_{1j} & \cdots & v_{1i} u_{1j} v_{ni} u_{nj} \\
                \vdots & \ddots & \vdots\\
                v_{ni} u_{nj} v_{1i} u_{1j} & \cdots & v_{ni} u_{nj} v_{ni}  u_{nj}
            \end{bmatrix}\\
            &= \sum_{i=1}^{n} \sum_{j=1}^{n} \lambda_{i} \mu_{j}
            \begin{bmatrix}
                v_{1i} u_{1j} \\ \vdots \\ v_{ni} u_{nj}
            \end{bmatrix}
            \begin{bmatrix}
                v_{1i} u_{1j} & \cdots & v_{ni} u_{nj}
            \end{bmatrix}\\
            &= \sum_{i=1}^{n} \sum_{j=1}^{n} \lambda_{i} \mu_{j}
            (\v_i \circ \u_j) (\v_i \circ \u_j)^\top,
        \end{align*}
        where \(\circ\) is the Hadamard product.
        Each \((\v_i \circ \u_j) (\v_i \circ \u_j)^\top\) is a symmetric positive semi-definite matrix.
        Since \(K_1, K_2\) are positive semi-definite, we have \(\lambda_i, \mu_i > 0\).
        Then \(K\) is symmetric positive semi-definite.
        \item By part \ref{itm:kernel-product}, \(k_1, k_1^2, \dots, k_1^n\) are kernels.
        By part \ref{itm:kernel-linear-combo}, \(a_0 + a_1 k_1 + a_2 k_1^2 + \dots + a_n k_1^n\) is a kernel.
        \item Since \(y_i = h(x_i) \in X\) for all \(i = 1,2,\dots, n\), we have
        \begin{align*}
            \sum_{i=1}^{n} \sum_{j=1}^{n} c_i c_j k(x_i,x_j)
            &= \sum_{i=1}^{n} \sum_{j=1}^{n} c_i c_j k_1(h(x_i), h(x_j))\\
            &= \sum_{i=1}^{n} \sum_{j=1}^{n} c_i c_j k_1(y_i, y_j)\\
            &\geq 0.
        \end{align*}
        \item Let \(g : X \to \RR\) and let \(c_i g(x_i) = y_i \in \RR\).
        If \(k(x,y) = g(x)g(y)\), then
        \begin{align*}
            \sum_{i=1}^{n} \sum_{j=1}^{n} c_i c_j k(x_i,x_j)
            &= \sum_{i=1}^{n} \sum_{j=1}^{n} c_i g(x_i) c_j g(x_j)\\
            &= \sum_{i=1}^{n} \sum_{j=1}^{n} y_i y_j\\
            % &= \sum_{\ell=1}^{n} \left(y_\ell^2 + \sum_{i=\ell+1}^{n} y_i y_\ell + \sum_{j=\ell+1}^{n} y_\ell y_j\right)\\
            % &= \sum_{i=1}^{n} \left(y_i^2 + 2\sum_{j=i+1}^{n} y_i y_j\right)\\
            &= \left(\sum_{i=1}^{n} y_i\right)^2\\
            &\geq 0.
        \end{align*}
        \item Let \(K_1\) be the Gram matrix for \(k_1\).
        If \(K_1 v = \lambda v\), then \(K_1^m = \lambda^m v\) for all \(m \in \NN\).
        So,
        \begin{align*}
            (\exp K_1) v
            = \sum_{m=0}^{\infty} \frac{K_1^m v}{m!}
            = \sum_{m=0}^{\infty} \frac{\lambda^m v}{m!}
            = e^\lambda v.
        \end{align*}
        Then \(K = \exp K_1\) has eigenvalues \(e^\lambda\).
        Since \(K_1\) is positive semi-definite, it has real eigenvalues so that \(e^\lambda > 0\).
        It follows that \(K\) is positive definite.
    \end{enumerate}
\end{proof}

\begin{theorem}[Gaussian kernel]
    \label{thm:gaussian-kernel}
    The function \(k : \RR^n \times \RR^n \to \RR\) defined by
    \[k(x,y) = \exp\left(\dfrac{-\|x-y\|^2_2}{\sigma^2}\right),\]
    is a kernel.
\end{theorem}

\begin{proof}
    
\end{proof}
\section{Kernel PCA}
\label{sec:kernel-pca}

PCA works by computing vector projections using the dot product.
This is the typical inner product for \(\RR^n\) and the resulting basis is orthogonal.
The idea behind kernel PCA is to replace the dot product with a kernel function.

\section{Conclusion}
\label{sec:conclusion}
Kernel PCA is a powerful tool that reveals nonlinear patterns in data.
This is only one example of an entire class of kernel methods that extend the capabilities of linear algorithms.
By exploring the theory behind kernel methods, we see how much structure exists due to symmetric positive semidefinite kernels.
\appendix
\section{Linear Algebra}
\label{sec:linear-algebra}
% \begin{theorem}[Properties of symmetric matrices]
%     \label{thm:properties-of-symmetric-matrices}
%     If \(A\) and \(B\) are symmetric, then the following properties hold:
%     \begin{enumerate}
%         \item \(A\) is normal, i.e., \(AA^\top = A^\top A\).
%         \item \(A\) is diagonalizable with real eigenvalues and orthogonal eigenvectors.
%         \item \(A^{-1}\) is symmetric.
%         \item \(A + B\) is symmetric.
%         \item The Hadamard product \(A \circ B\) is symmetric, where \([A \circ B]_{ij} = [A]_{ij} [B]_{ij}\).
%         \item \(ABA\) is symmetric.
%     \end{enumerate}
% \end{theorem}

\begin{lemma}
    Symmetric matrices have orthogonal eigenvectors and real eigenvalues.
\end{lemma}
\begin{proof}
    
\end{proof}

\begin{lemma}
    Positive semi-definite matrices have nonnegative eigenvalues.
\end{lemma}
\begin{proof}
    
\end{proof}

% \begin{lemma}
%     \label{lem:spsd-decomposition}
%     If \(A\) is symmetric positive semi-definite, then we can write
%     \[A = V^{\top} V.\]
% \end{lemma}
% \begin{proof}
%     \def\diag{\operatorname{diag}}
%     Since \(A\) is symmetric, it can be diagonalized by \(A = Q^{\top}DQ\), where \(Q\) is orthogonal and \(D = \diag(\lambda_1, \lambda_2, \dots, \lambda_n)\).
%     Since \(A\) is positive semi-definite, \(\lambda_1, \lambda_2, \dots, \lambda_n\) are nonnegative so that \(D^{1/2} = \diag\left(\sqrt{\lambda_1}, \sqrt{\lambda_2}, \dots, \sqrt{\lambda_n}\right)\).
%     Let \(V = D^{1/2} Q\).
%     Then
%     \[A = Q^{\top}DQ = Q^{\top} D^{1/2} D^{1/2} Q = \left(D^{1/2} Q\right)^{\top} \left(D^{1/2} Q\right) = V^{\top} V.\]
% \end{proof}

\begin{lemma}
    Let \(A\) be an \(m \times n\) matrix.
    Then \(A^\top A\) is symmetric positive semi-definite.
\end{lemma}

\begin{proof}
    First, \((A^\top A)^\top = A^\top (A^\top)^\top = A^\top A\).
    So \(A^\top A\) is symmetric.
\end{proof}
\begin{definition}
    \def\a{\mathbf{a}}
    Let \(A\) be an \(m \times n\) matrix whose columns \(\a_1, \a_2,\dots,\a_n \in \RR^m\) represent random variables and rows represent observations.
    Then the covariance matrix of \(A\) is given by
    \[\cov(A) = \begin{bmatrix}
        \cov(\a_1, \a_1) & \cov(\a_1, \a_2) & \cdots & \cov(\a_1, \a_n)\\
        \cov(\a_2, \a_1) & \cov(\a_2, \a_2) & \cdots & \cov(\a_2, \a_n)\\
        \vdots & \vdots & \ddots & \vdots\\
        \cov(\a_n, \a_1) & \cov(\a_n, \a_2) & \cdots & \cov(\a_n, \a_n)\\
    \end{bmatrix}.\]
    If \(A\) is centered, i.e., the columns all have mean zero, then we can write
    \[\cov(A) = \frac{A^\top A}{m-1}.\]
    % Here, \(1/(m-1)\) is due to Bessel's correction.
    % https://en.wikipedia.org/wiki/Bessel%27s_correction
\end{definition}

% \subsection{Schur product theorem}

% Let \(A, B \in \RR^{n \times n}\) where \([A]_{ij} = a_{ij}\) and \([B]_{ij} = b_{ij}\) for all \(i,j = 1,2,\dots,n\).

% \begin{lemma}
%     Let \(A\) be positive semi-definite with eigenvalues \(\lambda_\)
% \end{lemma}

% \begin{theorem}[Schur product theorem]
%     Let \(A\) and \(B\) be positive semi-definite matrices and let \(C = A \circ B\) be the Hadamard product of \(A\) and \(B\) given by \([C]_{ij} = [A]_{ij} [B]_{ij}\).
%     Then \(C\) is positive semi-definite.
% \end{theorem}

% \begin{proof}
    
% \end{proof}

% \section{Riesz Representation Theorem}
% \label{sec:riesz-representation-theorem}
% \begin{theorem}[Riesz Representation Theorem]
    \cite{small1994hilbert} % page 22
    Let \(\phi : H \to \RR\) be a continuous linear functional defined on a Hilbert space \(H\).
    Then there exists a unique element \(g \in H\) such that \(\phi(g) = \langle f, g \rangle_H\) for all \(g \in H\).
\end{theorem}

\begin{proof}
    
\end{proof}

% To do
% - linear functionals
% - construct RKHS
% \section{Mercer's Theorem}
% \label{sec:mercers-theorem}
% \input{sections/090-mercers-theorem.tex}
\section{Code}
\label{sec:code}
% TODO: Add code
\lstinputlisting[
    language=Python,
    caption=PCA example,
    label=lst:pca-example
]{../py/pca-example.py}
\lstinputlisting[
    language=Python,
    caption=PCA source functions,
    label=lst:pca-source
]{../py/pca.py}


% \nocite{*}
\bibliographystyle{plain}
\bibliography{refs}
\end{document}