
\begin{definition}[Definite matrix]
    \label{def:definite-matrix}
    \cite{horn2013matrix}
    Let \(A\) be an \(n \times n\) matrix over the real numbers having the form
\begin{equation}
    A = \begin{bmatrix}
        a_{11} & a_{12} & \cdots & a_{1n}\\
        a_{21} & a_{22} & \cdots & a_{2n}\\
        \vdots & \vdots & \ddots & \vdots\\
        a_{n1} & a_{n2} & \cdots & a_{nn}\\
    \end{bmatrix} = [a_{ij}].
\end{equation}
The \textit{transpose} of \(A\) is \(A^\top = [a_{ji}]\).
We say that \(A\) is \textit{symmetric} whenever \(A = A^\top\).
A symmetric matrix \(A\) is
\begin{enumerate}
    \item \label{part:positive-definite-matrix}
    \textit{positive definite} if \(x^\top A x > 0\), for all nonzero \(x \in \RR^n\), or
    \item \label{part:positive-semidefinite-matrix}
    \textit{positive semidefinite} if \(x^\top A x \geq 0\), for all \(x \in \RR^n\).
\end{enumerate} 
% \begin{align}
%     x^\top A x &> 0, \quad \text{for all nonzero \(x \in \RR^n\)}
%     \intertext{and is \textit{positive semidefinite} if}
%     x^\top A x &\geq 0, \quad \text{for all \(x \in \RR^n\)}.
% \end{align}
Negative (semi)definite matrices can be defined in a similar fashion.
A matrix is \textit{definite} if it is either positive semidefinite or negative semidefinite.
Otherwise, \(A\) is an \textit{indefinite matrix}.

Be aware that some authors use the terms positive definite (\(>\)) and nonnegative definite (\(\geq\)).
Other authors use the modifier \textit{strict} as in strict positive definite (\(>\)) and positive definite (\(\geq\)).
\end{definition}

\begin{definition}[Inner product]
    \label{def:inner-product}
    \cite{small1994hilbert} % page 10
    % begin macros
\def\innerprod{\ipt{\cdot, \cdot}}
\def\qand{\quad\text{and}\quad}
% end macros
% 
Let \(\innerprod : \X \times \X \to \RR\) be a function defined on the vector space \(\X\).
Then \(\innerprod\) is an \textit{inner product} if the following properties hold for all \(x, y, z \in \X\) and \(\alpha, \beta \in \RR\):
\begin{enumerate}
    \item %Symmetry:
    \(\ipt{x,y} = \ipt{y,x}\);
    \hfill (symmetry)
    \item %Bilinearity:
    \(\ipt{\alpha x + \beta y, z} = \alpha \ipt{x, z} + \beta \ipt{y, z}\);
    \hfill (bilinear)
    \item %Positive definiteness:
    \(\ipt{x,x} \geq 0\) and \(\ipt{x,x} = 0\) if and only if \(x = 0\).
    \hfill (positive definite)
\end{enumerate}
Note that linearity in the first argument with symmetry implies that the inner product is bilinear (linear in both arguments).
Since inner products are positive definite, they induce a norm \(\| \cdot \| : \X \to \RR\) and metric \(d : \X \times \X \to \RR\) such that
\begin{equation}
    \|x\| = \sqrt{\langle x, x \rangle}
    \qand
    d(x,y) = \|x - y\|.
\end{equation}
An \textit{inner product space}, \textit{normed space}, and \textit{metric space} are vector spaces along with an inner product, norm, and metric, respectively.
It follows that an inner product space is also a normed space and a metric space.
Then the induced norm has the following properties for all \(x, y \in \X\) and \(\alpha \in \RR\):
\begin{enumerate}
    \item \(\|\alpha x\| = |\alpha| \|x\|\);
    \item \(\|x\| \geq 0\) and \(\|x\| = 0\) if and only if \(x = 0\);
    \item \(\|x + y \| \leq \|x\| + \|y\|\);
    \hfill (triangle inequality)
    \item \(\langle x, y \rangle^2 \leq \|x\| \|y\|\).
    \hfill (Cauchy-Schwarz inequality)
\end{enumerate}
\end{definition}

\begin{example}
    \label{eg:dot-product}
    The \textit{Euclidean inner product} (or \textit{dot product}) is the function \(\langle \cdot, \cdot \rangle : \RR^n \times \RR^n \to \RR\) such that
\begin{equation}
    \langle x, y \rangle = \sum_{i=1}^{n} x_i y_i = x^\top y,
\end{equation}
for all \(x = [x_i], y = [y_i] \in \RR^n\).
Sometimes we write \(x \dotprod y\) to mean the Euclidean inner product.
This induces the \textit{Euclidean norm}
\begin{equation}
    \|x\| = \sqrt{\sum_{i=1}^{n} x_i^2}
\end{equation}

% Checking that this is an inner product:
% \begin{enumerate}
%     \item \(\begin{aligned}[t]
%         \langle x, y \rangle = \sum_{i=1}^{n} x_i y_i = \langle y, x \rangle
%     \end{aligned}\);
%     \item \(\begin{aligned}[t]
%         \langle \alpha x + \beta y, z \rangle
%         &= \sum_{i=1}^{n} (\alpha x_i + \beta y_i) z_i\\
%         &= \alpha \sum_{i=1}^{n} x_i z_i + \beta \sum_{i=1}^{n} y_i z_i
%         = \alpha \langle x, z \rangle
%         + \beta \langle y, z \rangle;
%     \end{aligned}\)
%     \item Let \(x \neq 0\).
%     Then there is some nonzero \(x_a \in \{x_1, x_2, \dots, x_n\}\) such that
%     \(\langle x, x \rangle = \sum_{i=1}^{n} x_i^2 \geq x_a^2 > 0\).
%     Also, \(\langle 0, 0 \rangle = 0\).
% \end{enumerate}
\end{example}

\begin{definition}[Hilbert space]
    \label{def:hilbert-space}
    \cite{kreyszig1991introductory}
    A metric space is \textit{complete} if the limit of every Cauchy sequence is in the space.
A complete normed space is called a \textit{Banach space}.
A complete inner product space \(\H\) is called a \textit{Hilbert space}.
We sometimes denote the inner product and norm of \(\H\) as \(\ipt{\cdot, \cdot}_H\) and \(\|\cdot\|_{\H}\) to avoid ambiguity.

% Kreyszig, Definition 1.3.5, page 21
A Hilbert space is said to be \textit{separable} if it contains a dense countable subset.
% Kreyszig, Theorem 3.6.4, page 171
It can be shown that a Hilbert space is separable if and only if it has a countable orthonormal basis.

% Kreyszig, page 173
\def\L{\hilbert{L}}
Two real Hilbert spaces \(\H\) and \(\L\) are said to be \textit{isomorphic} if there is a linear bijection \(T : \H \to \L\) such that
\begin{equation}
    \label{eqn:hilbert-space-isomorphism}
    \ipt{x,y}_{\H} = \ipt{Tx, Ty}_{\L},
\end{equation}
for every \(x, y \in \H\).
In \cite{kreyszig1991introductory}, Kreyszig shows that two Hilbert spaces are isomorphic if and only if they have the same dimension.
% Theorem 3.6.5, page 173
If \(\H\) is an inner product space that is not complete, then it can be extended to a Hilbert space by completion.
The completion of an inner product space is denoted as \(\overline{\H}\) and is unique up to isomorphism.
\end{definition}

\begin{theorem}[Properties of Hilbert spaces]
    \label{thm:hilbert-space-properties}
    \cite{kreyszig1991introductory}
    \def\L{\hilbert{L}}
\def\iptH#1{\ipt{#1}_{\H}}
\def\iptL#1{\ipt{#1}_{\L}}
Then the following properties hold for Hilbert spaces.
\begin{enumerate}
    \item A Hilbert space is separable if and only if it has a countable orthonormal basis.
    % Kreyszig, Theorem 3.6.4, page 171
    \item Two Hilbert spaces are isomorphic if and only if they have the same dimension.
    % Kreyszig, Theorem 3.6.5, page 173
    \item Any inner product \(\H\) space can be extended to a Hilbert space by completion and is unique up to isomorphism.
    We denote the completion as \(\overline{\H}\).
    \item A Hilbert space \(\H\) can be decomposed into orthogonal subspaces \(M\) and \(M^\perp\) such that whenever \(f \in M\) and \(g \in M^\perp\), then \(\ipt{f, g} = 0\)
    \cite{bachman1966functional}.
    We denote the \emph{orthogonal decomposition} of a Hilbert space as the direct sum \(\H = M \oplus M^\perp\).
\end{enumerate}
\end{theorem}

The following example demonstrates a useful property of separable Hilbert spaces.

\begin{example}
    \label{eg-l2-hilbert-space}
    \cite{rudin1987real}
    % Rudin, pages 84-85
Let \(A\) be a nonempty index set.
The space of square-summable indexed families is defined as
\begin{equation}
    \label{eqn:l2-space}
    \ell^2(A) = \left\{
        x : A \to \RR \;\middle\vert\; \sum_{a \in A} x_a^2 < \infty
    \right\}.
\end{equation}
Given the inner product
\begin{equation}
    \label{eqn:l2-inner-product}
    \langle x, y \rangle = \sum_{a \in A} x_a y_a,
\end{equation}
\(\ell^2(A)\) is a Hilbert space.
Moreover, \(\ell^2(A)\) is separable if and only if \(A\) is countable.
It follows that the sequence space \(\ell^2 = \ell^2(\NN)\) is the separable Hilbert space of square-summable sequences.
Due to the Riesz-Fischer theorem, every infinite-dimensional Hilbert space is isomorphic to \(\ell^2\).
\end{example}

% TODO: Define Hilbert space dual

\begin{definition}[Gram matrix]
    \label{def:gram-matrix}
    \cite{horn2013matrix}
    Let \(x_1, x_2, \dots, x_n \in \X\) for some inner product space \(\X\) equipped with \(\langle \cdot, \cdot \rangle\).
We say \(G\) is a \textit{Gram matrix} (or \textit{Gramian}) for the sequence of vectors \(x_1, x_2, \dots, x_n\) with respect to \(\langle \cdot, \cdot \rangle\) if \(G = \begin{bmatrix}
    \langle x_i, x_j \rangle
\end{bmatrix}_{ij}\).
\end{definition}

\begin{example}
    \label{eg:gram-matrix}
    % TODO: Clean up gram matrix example.

% begin macros
\def\v{\mathbf{v}}
% end macros
% 
Consider the vectors in \(\RR^3\):
\begin{align*}
    \v_1 &=
    \begin{bmatrix}
        v_{11} \\ v_{21} \\ v_{31}
    \end{bmatrix},&
    \v_2 &=
    \begin{bmatrix}
        v_{12} \\ v_{22} \\ v_{32}
    \end{bmatrix},&
    \v_3 &=
    \begin{bmatrix}
        v_{13} \\ v_{23} \\ v_{33}
    \end{bmatrix},&
    \v_4 &=
    \begin{bmatrix}
        v_{14} \\ v_{24} \\ v_{34}
    \end{bmatrix}.
\end{align*}
The Gram matrix for these vectors is
\begin{align*}
    G = \begin{bmatrix}
        \v_1^\top \v_1 & \v_1^\top \v_2 & \v_1^\top \v_3 & \v_1^\top \v_4\\[3pt]
        \v_2^\top \v_1 & \v_2^\top \v_2 & \v_2^\top \v_3 & \v_2^\top \v_4\\[3pt]
        \v_3^\top \v_1 & \v_3^\top \v_2 & \v_3^\top \v_3 & \v_3^\top \v_4\\[3pt]
        \v_4^\top \v_1 & \v_4^\top \v_2 & \v_4^\top \v_3 & \v_4^\top \v_4\\
    \end{bmatrix}.
\end{align*}
If \(V\) is a matrix whose columns are \(\v_1\), \(\v_2\), \(\v_3\), \(\v_4\), then we can write \(G = V^\top V\).
\end{example}

\begin{theorem}
    \label{thm:gram-psd}
    \cite{horn2013matrix}
    A matrix \(G\) is a Gram matrix if and only if \(G\) is positive semidefinite.
\end{theorem}
\begin{proof}
    (\(\Rightarrow\))
Suppose \(G\) is the Gram matrix of \(x_1, x_2, \dots, x_n\) with respect to \(\langle \cdot, \cdot \rangle\).
Let \(c_1, c_2, \dots, c_n \in \RR\).
Then \(G\) is positive semidefinite because
\begin{align}
    \sum_{i=1}^{n} \sum_{j=1}^{n} c_i c_j \langle x_i, x_j \rangle
    = \left\langle
        \sum_{i=1}^{n} c_i x_i, \sum_{j=1}^{n} c_j x_j
    \right\rangle
    = \left\|
        \sum_{i=1}^{n} c_i x_i
    \right\|^2
    \geq 0.
\end{align}

(\(\Leftarrow\))
Suppose \(G\) is positive semidefinite.
Then \(G\) can be factored as \(G = B^\top B\).
Let \(b_1, b_2, \dots, b_n\), be the columns of \(B\).
Then \(G = [b_i^\top b_j]_{ij}\).
Hence \(G\) is the Gram matrix of \(b_1, b_2, \dots, b_n\) with respect to the dot product.
\end{proof}

% \begin{theorem}
%     \label{thm:gram-pd-iff-lin-indep}
%     \cite{horn2013matrix}
%     A Gram matrix \(G\) of \(x_1, x_2, \dots, x_n\) is positive definite if and only if \(x_1, x_2, \dots, x_n\) are linearly independent.
% \end{theorem}
% \begin{proof}
%     % TODO: Prove that a Gram matrix is positive semidefinite iff its columns are linearly independent.

% This proof is in Horn

% \end{proof}

\begin{definition}[Symmetric bilinear form]
    \label{def:symmetric-bilinear-form}
    A \textit{symmetric bilinear form} is a map \(k : \X \times \X \to \RR\) over a vector space \(\X\) such that, for all \(x,y,z \in \X\), \(\alpha, \beta \in \RR\),
\begin{enumerate}
    \item \(k(x,y) = k(y,x)\) and
    \hfill (symmetry)
    \item \(k(\alpha x + \beta y, z) = \alpha k(x,z) + \beta k(y,z)\).
    \hfill (bilinear)
\end{enumerate}
\end{definition}

This can be thought of as a generalization of an inner product which is symmetric and bilinear, but not necessarily positive definite.

\begin{example}
    \label{eg:symmetric-bilinear-form}
    If \(U = \{u_1, u_2, \dots, u_n\}\) is a basis for \(\X\), then we can define a matrix \(K = \begin{bmatrix}
    k(u_i, u_j)
\end{bmatrix}_{ij}\).
Clearly, \(K\) is symmetric since \(k(u_i, u_j) = k(u_j, u_i)\).
Let \(v = \sum_{i=1}^n \alpha_i u_i\) and \(w = \sum_{i=1}^n \beta_i u_i\) be vectors with respect to \(U\) and let \(x = [\alpha_i]_{i=1}^n\) and \(y = [\beta_i]_{i=1}^n\).
Then
\begin{equation}
    \label{eqn:bilinear-form-representation}
    k(v,w)
    = k\left(\sum_{i=1}^n \alpha_i u_i, \sum_{j=1}^{n} \beta_j u_j\right)
    = \sum_{i=1}^n \sum_{j=1}^{n} \alpha_i \beta_j k\left(u_i, u_j\right)
    = x^\top K y.
\end{equation}
If \(K = I\), then \(v = x\), \(w = y\), and \(k(v,w) = v^\top w\) is simply the dot product.
Otherwise, if \(K\) is positive semidefinite, then \(K = B^\top B\) implies
\begin{equation}
    \label{eqn:bilinear-form-dot-product}
    k(v,w) = x^\top B^\top B y = (Bx)^\top (By)
\end{equation}
In this case, \(k(v,w)\) is just the dot product after the transformation under \(B\).
Notice that if \(U\) is merely a subset of \(\X\), then \(v\) and \(w\) no longer have unique representations, but \cref{eqn:bilinear-form-representation,eqn:bilinear-form-dot-product} are still valid for all \(v, w \in \lspan U\).

We say that \(k\) is \textit{positive semidefinite} if \(K = [k(u_i, u_j)]_{ij}\) is a positive semidefinite matrix for any finite subset \(U = \{u_1, u_2,\dots, u_n\} \subseteq \X\).
Then \(K\) is a Gram matrix with respect to some set of transformed vectors related to \(U\) and some inner product related to \(k\).
In the next subsection, we will show that \(k\) still corresponds to some inner product even if \(k\) is not bilinear.

\end{example}