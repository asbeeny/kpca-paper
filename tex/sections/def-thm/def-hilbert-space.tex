A metric space is \textit{complete} if the limit of every Cauchy sequence is in the space.
A complete normed space is called a \textit{Banach space}.
A complete inner product space \(\H\) is called a \textit{Hilbert space}.
We sometimes denote the inner product and norm of \(\H\) as \(\ipt{\cdot, \cdot}_H\) and \(\|\cdot\|_{\H}\) to avoid ambiguity.

% Kreyszig, Definition 1.3.5, page 21
A Hilbert space is said to be \textit{separable} if it contains a dense countable subset.
% Kreyszig, Theorem 3.6.4, page 171
It can be shown that a Hilbert space is separable if and only if it has a countable orthonormal basis.

% Kreyszig, page 173
\def\L{\hilbert{L}}
Two real Hilbert spaces \(\H\) and \(\L\) are said to be \textit{isomorphic} if there is a linear bijection \(T : \H \to \L\) such that
\begin{equation}
    \label{eqn:hilbert-space-isomorphism}
    \ipt{x,y}_{\H} = \ipt{Tx, Ty}_{\L},
\end{equation}
for every \(x, y \in \H\).
In \cite{kreyszig1991introductory}, Kreyszig shows that two Hilbert spaces are isomorphic if and only if they have the same dimension.
% Theorem 3.6.5, page 173
If \(\H\) is an inner product space that is not complete, then it can be extended to a Hilbert space by completion.
The completion of an inner product space is denoted as \(\overline{\H}\) and is unique up to isomorphism.