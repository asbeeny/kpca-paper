% begin macros
\def\v{v}
% end macros
% 
Consider the vectors in \(\RR^3\):
\begin{align*}
    \v_1 &=
    \begin{bmatrix}
        v_{11} \\ v_{21} \\ v_{31}
    \end{bmatrix},&
    \v_2 &=
    \begin{bmatrix}
        v_{12} \\ v_{22} \\ v_{32}
    \end{bmatrix},&
    \v_3 &=
    \begin{bmatrix}
        v_{13} \\ v_{23} \\ v_{33}
    \end{bmatrix},&
    \v_4 &=
    \begin{bmatrix}
        v_{14} \\ v_{24} \\ v_{34}
    \end{bmatrix}.
\end{align*}
The Gram matrix for these vectors is
\begin{align*}
    G = \begin{bmatrix}
        \v_1^\top \v_1 & \v_1^\top \v_2 & \v_1^\top \v_3 & \v_1^\top \v_4\\[3pt]
        \v_2^\top \v_1 & \v_2^\top \v_2 & \v_2^\top \v_3 & \v_2^\top \v_4\\[3pt]
        \v_3^\top \v_1 & \v_3^\top \v_2 & \v_3^\top \v_3 & \v_3^\top \v_4\\[3pt]
        \v_4^\top \v_1 & \v_4^\top \v_2 & \v_4^\top \v_3 & \v_4^\top \v_4\\
    \end{bmatrix}.
\end{align*}
If \(V\) is a matrix whose columns are \(\v_1\), \(\v_2\), \(\v_3\), \(\v_4\), then we can write \(G = V^\top V\).