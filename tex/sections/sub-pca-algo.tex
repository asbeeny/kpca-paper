Let \(A\) be a data matrix whose \(n\) rows correspond to observations and \(d\) columns correspond to variables.
The following algorithm demonstrates a simple method for computing the PCA of \(A\):
\begin{enumerate}
    \item Compute the centered matrix \(A_0 = A - \colmean(A)\).
    \item Compute the covariance matrix \(C = \frac{1}{n-1} A_0^\top A_0\).
    \item Diagonalize the covariance matrix such that \(C = V D V^\top\).
    \item Order the eigenvalues and eigenvectors so that \(\lambda_1 \geq \lambda_2 \geq \cdots \geq \lambda_d\).
    We call the ordered eigenvalues the \textit{principal components}.
    \item Choose the dimension of the subspace \(p \leq d\).
    \item Construct the rank-\(p\) transformation matrix \(V_p \in \RR^{d \times p}\) using the first \(p\) principal components \(v_1, v_2, \dots, v_p\).
    \item The image of \(A\) under the PCA transform is \(B = A_0 V_p\).
\end{enumerate}