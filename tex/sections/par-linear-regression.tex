In linear regression, the equation of a line \(y = a_0 + a_1 x\) is used to model observations based on training data.
Here, the input variable \(x\) is used to predict the response variable \(y\).
The parameters \(a_0\) and \(a_1\) are chosen such that the residual error\footnote{The residual for a given observation \(\left(x^{(i)},y^{(i)}\right)\) is \(\left|y^{(i)} - \left(a_0 + a_1 x^{(i)}\right)\right|\).} is minimized.
It seems natural to model data using polyomial equations in a similar way, that is, determine \(a_0, a_1, \dots, a_n\) such that
\begin{equation}
    \label{eqn:polynomial-regression}
    y = a_0 + a_1 x + a_2 x^2 + \dots + a_n x^n
\end{equation}
minimizes the residual error.

In multiple linear regression, the equation of a hyperplane
\begin{equation}
    \label{eqn:multiple-linear-regression}
    y = a_0 + a_1 x_1 + a_2 x_2 + \dots + a_n x_n
\end{equation}
is used to predict \(y\) using inputs \(x_1, x_2, \dots x_n\).
It follows that these observations are points in \(n + 1\) dimensions.