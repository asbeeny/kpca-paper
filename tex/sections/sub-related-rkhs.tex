In the previous section, we showed that a (symmetric positive semidefinite) kernel defines an RKHS.
Presently, we will look at three alternative methods for constructing an RKHS.
\begin{enumerate}
    \item \textbf{Positive semidefinite kernels.}
    Due to Aronszajn \cite{aronszajn1950theory}, a reproducing kernel will generate a unique RKHS.
    Moreover, a kernel is unique to its RKHS.
    See \Cref{thm:moore-aronszajn,thm:rk-uniqueness}.
    \item \textbf{Continuous linear functionals.}
    By the Riesz representation theorem, if every evaluation functional is continuous, then every function in the Hilbert space can be reproduced at every point.
    In this way, a reproducing kernel can be defined.
    \item \textbf{Feature maps.}
    A explicit feature map with an inner product can be used to define a kernel as in \cref{eqn:kernel-inner-product-1}.
    Alternatively, by Mercer's theorem \ref{thm:mercers}, a kernel has a series expansion which allows us to define a feature map in terms of eigenvalues and eigenfunctions.
\end{enumerate}

\subsubsection{Positive semidefinite kernels.}

\begin{theorem}[Moore-Aronszajn theorem]
    \label{thm:moore-aronszajn}
    \cite{aronszajn1950theory}
    Let \(\X\) be a nonempty set and \(k : \X \times \X \to \RR\) be a kernel.
Then there exists a unique RKHS for which \(k\) is a reproducing kernel.
\end{theorem}
\begin{proof}
    % Existence
For existence, we summarize the construction provided in \Cref{sub:kernels}.
\begin{enumerate}
    \item \(k\) defines a feature map \(\Phi : \X \to \RR^\X\) such that \(\Phi(x) = k(x, \cdot)\) for all \(x \in \X\).
    \item The linear span of \(\Phi(\X)\) is a feature space.
    \item \(\Phi\) defines an inner product \(\iptHo{\cdot, \cdot} : \Ho \times \Ho \to \RR\) in \cref{eqn:pre-hilbert-inner-product}.
    \item Completing the feature space yields a Hilbert space \(\H\) with inner product \(\ipt{\cdot, \cdot} : \H \times \H \to \RR\).
    \item \(k\) has the reproducing property \(\ipt{f, k(x,\cdot)} = f(x)\) shown by \cref{eqn:reproducing-property}.
\end{enumerate}
\def\L{\hilbert{L}}
\def\Hp{\H^{\perp}}
% Uniqueness
For uniqueness, suppose \(k\) is a reproducing kernel for Hilbert spaces \(\H\) and \(\L\).

TODO: Show that WLOG \(\H \subseteq \L\).

Now let \(f \in \L\).
Then we can write \(\L = \H \oplus \Hp\) as the orthogonal decomposition of \(\L\).
There exist \(g \in \H\) and \(g^\perp \in \Hp\) such that \(f = g + g^\perp\).
Let \(x \in \X\).
Then \(k(x,\cdot) \in \H\) implies \(\ipt{g^\perp,k(x,\cdot)}_{\L} = 0\)
Thus
\begin{equation}
    f(x)
    = \ipt{g,k(x,\cdot)}_{\L} + \ipt{g^\perp,k(x,\cdot)}_{\L}
    = \ipt{g,k(x,\cdot)}_{\L}
    = \ipt{g,k(x,\cdot)}_{\H}
    = g(x).
\end{equation}
Thus \(f = g \in \H\) implies \(\L \subseteq \H\).
\end{proof}

\begin{theorem}
    \label{thm:rk-uniqueness}
    \cite{aronszajn1950theory}
    A reproducing kernel for an RKHS is unique.
\end{theorem}
\begin{proof}
    Let \(\H\) be an RKHS of functions \(\X \to \RR\) for some set \(\X \neq \emptyset\).
Suppose \(k\) and \(\ell\) reproducing kernels for \(\H\).
Denote \(k_x = k(x,\cdot)\) and \(\ell_x = \ell(x, \cdot)\), for all \(x \in \X\).
By the reproducing property,
\begin{align}
    \|k_x - \ell_x\|^2
    &= \ipt{k_x - \ell_x, k_x - \ell_x}\\
    \notag
    &= \ipt{k_x - \ell_x, k_x} - \ipt{k_x - \ell_x, \ell_x}\\
    \notag
    &= k_x(x) - \ell_x(x) - k_x(x) + \ell_x(x)\\
    \notag
    &= 0.
\end{align}
It follows that \(k_x - \ell_x\) is the zero function.
Hence, \(k(x,\cdot) = \ell(x,\cdot)\) for all \(x \in \X\).
By symmetry, \(k(\cdot, x) = \ell(\cdot, x)\).
Therefore, \(k = \ell\).
\end{proof}

\subsubsection{Continuous linear functionals.}

Suppose we have a Hilbert space and we want to know if it is an RKHS.
To do this, we need to construct a kernel from the Hilbert space.

\begin{example}
    \label{eg:r4-rkhs}
    Consider a Hilbert space \(\RR^n\) with the dot product.
Note that the vectors in \(\RR^n\) are actually just sequences \(\NN \to \RR\) with some vector operations.
It is straightforward to show that the reproducing kernel is the Kroenecker delta \(k(i,j) = \delta_{ij}\) for all \(i,j \in \{1,2,\dots,n\}\).
This generates the standard basis \(\{e_i\}_{i=1}^n\), where \(e_i = k(i,\cdot)\).
So, \(\RR^n\) is an RKHS.

We can replace \(\NN\) with any other index set with cardinality \(n\), say \(I_n = \{\frac{0}{n}, \frac{1}{n}, \frac{2}{n}, \dots, \frac{n-1}{n-1}\}\).
Then an indexed family \(f_n : I_n \to \RR\) has a vector representation in \(\RR^n\).
Letting \(n\) tend to infinity, we have \(I_\infty = \QQ \cap [0,1]\).
By completion, this is the space of functions \(\{f \mid f : [0,1] \to \RR\}\).
In one sense, \(f_n \in \RR^n\) is a point in \(n\)-dimensional space and, in another sense, \(f_n : I_n \to \RR\) is the discretization of a function \(f : [0,1] \to \RR\).
This way, real-valued functions on \([0,1]\) can be interpreted as infinite-dimensional vectors.

Now consider the Hilbert space \(L^2({[0,1]}) = \{f \mid \int_{[0,1]} f^2 < \infty\}\) with inner product \(\ipt{f,g} = \int_{[0,1]} fg\).
Then for all \(x \in {[0,1]}\),
\def\dd{\operatorname{d}}
\begin{equation}
    f(x) = \int_{[0,1]} \delta(x-t) f(t) \dd{t},
\end{equation}
where \(\delta\) is the Dirac delta function.
If \(L^2({[0,1]})\) is an RKHS, then by \Cref{thm:rk-uniqueness}, \(k(x,t) = \delta(x-t)\) is the unique reproducing kernel.
But \(\int_\X \delta^2 = \infty\) implies \(\delta \notin L^2({[0,1]})\).
Therefore, \(L^2({[0,1]})\) is not an RKHS.
\end{example}

The Kroenecker delta and Dirac delta in the example reproduce functions in the Hilbert space with the inner product.

\begin{definition}[Evaluation functional]
    \label{def:eval-functional}
    Let \(\X\) be a nonempty set and \(\H\) be a Hilbert space of functions \(\X \to \RR\).
    Then for all \(x \in \X\), let \(\delta_x : \H \to \RR\) such that \(\delta_x(f) = f(x)\), for each \(f \in \H\).
    We call \(\delta_x\) the \textit{evaluation functional} at \(x\).
\end{definition}

\begin{theorem}[Riesz representation theorem]
    \cite{small1994hilbert} % page 22
    Let \(\delta : H \to \RR\) be a continuous linear functional defined on a Hilbert space \(\H\).
    Then there exists a unique element \(g \in \H\) such that \(\delta(g) = \langle f, g \rangle_H\) for all \(g \in \H\).
\end{theorem}

Suppose the evaluation functional \(\delta_x\) is continuous on \(\H\) for every \(x \in \X\).
If \(x,y \in \X\), then, by the Riesz representation theorem, there exist \(k_x, k_y \in \H\) such that for all \(f \in \H\),
\begin{align}
    f(x) &= \delta_x(f) = \ipt{f, k_x},&
    f(y) &= \delta_y(f) = \ipt{f, k_y}.
\end{align}
Let \(k : \X \times \X \to \RR\) be defined as \(k(x,y) = k_x(y)\).
Then by symmetry of the inner product,
\begin{align}
    k(x,y) = k_x(y) = \ipt{k_x, k_y} = \ipt{k_y, k_x} = k_y(x) = k(y,x).
\end{align}
It follows that \(k\) is a kernel, \(\Phi(x) = k_x\) is a feature map, and \(\H\) is an RKHS.
% \begin{proof}
    
% \end{proof}

\subsubsection{Feature maps}
\label{sub:feature-maps}
Mercer's theorem provides a result similar to Aronszajn's, but without the context of an RKHS.
Rather, the focus is the decomposition of a kernel into a uniformly convergent series of eigenvalues and eigenfunctions.
This allows us to write a power series representation for the feature map.

% todo: remove mercer's theorem and riesz representation from appendix
\begin{theorem}[Mercer's theorem]
    \label{thm:mercers}
    \cite{mercer1909xvi}
    Let \(k : \X \times \X \to \RR\) be a continuous bounded kernel on a compact set \(\X\).
Define the Hilbert-Schmidt integral operator \(T_k : L^2(\X) \to L^2(\X)\) as
\begin{equation}
    (T_k f)(x) = \int_{\X} k(x, t) f(t) \dd t.
\end{equation}
% Then \emph{Mercer's condition} is satisfied
% \begin{equation}
%     \iint k(x,t) f(x) f(t) \dd x \dd t \geq 0
% \end{equation}
Then there exists an orthonormal basis \(\{\psi_{i}\}_{i=1}^\infty\) of eigenfunctions of \(T_k\) and corresponding eigenvalues \((\lambda_i)_{i=1}^\infty\) with \(\lambda_i \geq 0\), for all \(i \in \NN\).
Moreover, for all \(x,y \in \X\),
\begin{equation}
    \label{eqn:kernel-decomposition}
    k(x,y) = \sum_{i=1}^{\infty} \lambda_i \psi_i(x) \psi_i(y),
\end{equation}
where convergence is uniform.
\end{theorem}
\begin{proof}
    See \cite{ghojogh2021reproducing} for a sketch of this proof.
Otherwise, this is main result proved in \cite{mercer1909xvi}.
\end{proof}