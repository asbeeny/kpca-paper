
Let \(\Phi : \X \to \H_k\) be a feature map determined by a kernel \(k\).
Since \(\Phi\) may be nonlinear, the image \(\Phi(x)\) of a centered vector \(x \in \X\) is not guaranteed to be centered.
For an effective PCA algorithm, it is necessary to compute the kernel matrix of centered vectors in the feature space. \cite{scholkopf1998nonlinear}

Given \(x_1, \dots, x_n \in \X\), the points
\begin{equation}
    \label{eqn:centered-features}
    \Phi_0(x_i) = \Phi(x_i) - \frac{1}{n} \sum_{i=1}^{n} \Phi(x_i), \quad \text{for \(i = 1,\dots,m\)}
\end{equation}
are the centered feature vectors in \(H_k\).
Then the centered kernel matrix becomes
\def\ipt#1{\left\langle #1 \right\rangle}
\begin{align*}
    [K_0]_{ij}
    &= \ipt{\Phi_0(x_i), \Phi_0(x_j)}\\
    &= \ipt{\Phi(x_i) - \frac{1}{n} \sum_{p=1}^{n} \Phi(x_p), \Phi(x_j) - \frac{1}{n} \sum_{q=1}^{n} \Phi(x_q)}\\
    &= \ipt{\Phi(x_i), \Phi(x_j)}
    \begin{aligned}[t]
        &- \frac{1}{n} \sum_{p=1}^{n} \ipt{\Phi(x_p), \Phi(x_j)}\\
        &- \frac{1}{n} \sum_{q=1}^{n} \ipt{\Phi(x_i), \Phi(x_q)}\\
        &+ \frac{1}{n^2} \sum_{p=1}^{n} \sum_{q=1}^{n} \ipt{\Phi(x_p), \Phi(x_q)}
    \end{aligned}\\
    % &= k(x_i, x_j) - \frac{1}{n} \sum_{p=1}^{n} k(x_p, x_j) - \frac{1}{n} \sum_{q=1}^{n} k(x_i, x_q) + \frac{1}{n^2} \sum_{p=1}^{n} \sum_{q=1}^{n} k(x_p, x_q)\\
    &= [K]_{ij} - \frac{1}{n} \sum_{p=1}^{n} [K]_{pj} - \frac{1}{n} \sum_{q=1}^{n} [K]_{iq} + \frac{1}{n^2} \sum_{p=1}^{n} \sum_{q=1}^{n} [K]_{pq},
\end{align*}
where \(K\) is the uncentered kernel matrix given by \([K]_{ij} = \ipt{\Phi(x_i), \Phi(x_j)}\).
% The term \(\frac{1}{n} \sum_{p=1}^{n} [K]_{pj}\) is the mean of the \(j\)-th column of \(K\), \(\frac{1}{n} \sum_{q=1}^{n} [K]_{iq}\) is the mean of the \(i\)-th row of \(K\), and \(\frac{1}{n^2} \sum_{p=1}^{n} \sum_{q=1}^{n} [K]_{pq}\) is the mean over all entries of \(K\).
Then the formula for the centered kernel matrix can be written as
\begin{equation}
    \label{eqn:centered-kernel-matrix}
    K_0 = K - \colmean(K) - \rowmean(K) + \mean(K).
\end{equation}
See notes in \Cref{subsec:matrix-operations,subsec:broadcasting}.
% \def\ones{\mathbf{1}_n}
% \begin{equation}
%     \label{eqn:centered-kernel-matrix}
%     K_0 = K - \ones K - K \ones + \ones K \ones,
% \end{equation}
% where \(\ones\) is the \(n \times n\) matrix whose entries are \(1/n\).
% So, the centered matrix \(K_0\) can be computed in terms of the kernel where \([K]_{ij} = k(x_i,x_j)\).

% \def\c{\mathbf{c}}
% \def\r{\mathbf{r}}
% For efficiency, the column means, row means, and entrywise mean of \(K\) can be precomputed and broadcast as \(m \times m\) arrays.
% Let
% \begin{align*}
%     \c &= \frac{1}{n} \begin{bmatrix}
%         \sum_{p=1}^{n} [K]_{p1}\\[4pt]
%         \sum_{p=1}^{n} [K]_{p2}\\[4pt]
%         \vdots\\[4pt]
%         \sum_{p=1}^{n} [K]_{pm}
%     \end{bmatrix}^\top,&
%     \r &= \frac{1}{n} \begin{bmatrix}
%         \sum_{q=1}^{n} [K]_{1q}\\[4pt]
%         \sum_{q=1}^{n} [K]_{2q}\\[4pt]
%         \vdots\\[4pt]
%         \sum_{q=1}^{n} [K]_{mq}
%     \end{bmatrix},&
%     \mu &= \frac{1}{n^2} \sum_{p=1}^{n} \sum_{q=1}^{n} [K]_{pq}.
% \end{align*}
% Then \cref{eqn:centered-kernel-matrix} becomes
% \begin{equation}
%     \label{eqn:centered-kernel-computation}
%     K_0 = K - \c - \r + \mu,
% \end{equation}
% where \(\c\) is a \(1 \times m\) array broadcast to the rows of \(K\), \(\r\) is an \(m \times 1\) array broadcast to the columns of \(K\), and \(\mu\) is a scalar added to every entry of \(K\).