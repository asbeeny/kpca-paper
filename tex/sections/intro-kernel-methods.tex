The development of \textit{kernel functions} can be traced back to the beginning of the twentieth century when David Hilbert and James Mercer were studying integral equations \cite{hofmann2008kernel}.
Hilbert proved some important results in \cite{hilbert1912grundzüge} about the eigenvalues of an integral operator whose kernel function is of \textit{definite} type.
Expanding on Hilbert's work, Mercer provided the necessary conditions in \cite{mercer1909xvi} that allow a kernel function to be written in terms of the eigenvalues and eigenfunctions of the integral operator.
This result became known as Mercer's theorem.
See \Cref{sec:mercers-theorem}.
A simplified version of Mercer's theorem states that a kernel function can be written as an inner product in a higher-dimensional space.

Hilbert space theory and Mercer's theorem led to a number of advances in functional analysis over the next few decades.
Notably, in 1950, Nachman Aronszajn introduced reproducing kernel Hilbert spaces in \cite{aronszajn1950theory}.
This work expanded on Mercer's theorem and shows that a kernel generates a Hilbert space whose inner product agrees with the kernel.

Later, the work of Mercer and Aronszajn inspired the application of kernels in machine learning.
A \textit{kernel method} is an adaptation of a machine learning algorithm that replaces a dot product with a kernel function.
The earliest research involving kernel methods was in 1964 by Mark Aizerman et al. \cite{aizerman1964theoretical}.
In the 1990s, Bernhard Sch\"olkopf et al. used Aizerman's technique to develop kernel PCA and suggested the kernel trick could work in other cases too.
In \Cref{sec:kernel-pca}, we will look at the kernel method applied to the PCA algorithm.
For now, we will examine the mathematics behind kernel methods.