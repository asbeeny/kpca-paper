In the previous section, we showed that a (symmetric positive semidefinite) kernel defines an RKHS.
Presently, we will look at three alternative methods for constructing an RKHS.
\begin{enumerate}
    \item \textbf{Positive semidefinite kernels.}
    Due to Aronszajn \cite{aronszajn1950theory}, a reproducing kernel will generate a unique RKHS.
    Moreover, a kernel is unique to its RKHS.
    See \Cref{thm:moore-aronszajn,thm:rk-uniqueness}.
    \item \textbf{Continuous linear functionals.}
    By the Riesz representation theorem, if every evaluation functional is continuous, then every function in the Hilbert space can be reproduced at every point.
    In this way, a reproducing kernel can be defined.
    \item \textbf{Feature maps.}
    A explicit feature map with an inner product can be used to define a kernel as in \cref{eqn:kernel-inner-product-1}.
    Alternatively, by Mercer's theorem \ref{thm:mercers}, a kernel has a series expansion which allows us to define a feature map in terms of eigenvalues and eigenfunctions.
\end{enumerate}