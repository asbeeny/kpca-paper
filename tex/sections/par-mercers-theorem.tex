\def\dd{\;\mathrm{d}}
\def\dx{\dd{x}}
\def\dy{\dd{y}}
\def\dt{\dd{t}}
Mercer's theorem provides a result similar to Aronszajn's, but without the context of an RKHS.
Rather, the focus is the decomposition of an integral operator.

\begin{theorem}[Mercer's theorem]
    \label{thm:mercers}
    \cite{mercer1909xvi}
    % todo: write mercer's theorem
\end{theorem}

See \cite{ghojogh2021reproducing} for a sketch of this proof.
Otherwise, this is main result proved in \cite{mercer1909xvi}.
    
Using \cref{eqn:kernel-decomposition}, we can define a feature map \(\Phi : \X \to \H\) such that for all \(x \in \X\),
\begin{equation}
    \label{eqn:mercers-feature-map}
    \Phi(x) = \sum_{i=1}^{\infty} \sqrt{\lambda_i} \psi_i(x).
\end{equation}
In the finite-dimensional case, we have \(\Phi(x) = \left[\sqrt{\lambda_j} \psi_j(x)\right]_{j=1}^d\).
It follows that \(k(x,y) = \ipt{\Phi(x), \Phi(y)}\).