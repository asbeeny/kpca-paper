Mercer's theorem provides a result similar to Aronszajn's, but without the context of an RKHS.
Rather, the focus is the decomposition of a kernel into a uniformly convergent series of eigenvalues and eigenfunctions.
This allows us to write a power series representation for the feature map.

% todo: remove mercer's theorem and riesz representation from appendix
\begin{theorem}[Mercer's theorem]
    \label{thm:mercers}
    \cite{mercer1909xvi}
    % todo: write mercer's theorem
\end{theorem}
\begin{proof}
    See \cite{ghojogh2021reproducing} for a sketch of this proof.
Otherwise, this is main result proved in \cite{mercer1909xvi}.
\end{proof}