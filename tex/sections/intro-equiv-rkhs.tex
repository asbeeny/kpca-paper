In the previous section, we showed that a (symmetric positive semidefinite) kernel defines an RKHS.
Presently, we will look at three alternative definitions for RKHS stated as equivalence theorems.
\begin{enumerate}
    \item \textbf{Positive semidefinite kernels.}
    Due to Aronszajn \cite{aronszajn1950theory}, a reproducing kernel will generate a unique RKHS.
    Moreover, a kernel is unique to its RKHS.
    See \Cref{thm:moore-aronszajn,thm:rk-uniqueness}.
    \item \textbf{Continuous linear functionals.}
    By the Riesz representation theorem, if every evaluation functional is continuous, then every function in the Hilbert space can be reproduced at every point.
    In this way, a kernel can be defined.
    \item \textbf{Feature maps.}
    A feature map together with an inner product can be used to define evaluation functionals.
    Demonstrating that these are continuous will satisfy the hypotheses of the Riesz representation theorem.
\end{enumerate}

